% $date: 2015-10-14 # среда

\section*{Cherchez la racine}
% /ʁasin/ https://ru.wiktionary.org/wiki/racine

% $authors:
% - Глеб Погудин

\begin{problems}

\item
Существует~ли функция, график которой на~координатной плоскости имеет общую
точку с~любой прямой?

\item
Докажите, что при умножении многочлена $(x + 1)^{n-1}$ на~любой многочлен,
отличный от~нуля, получается многочлен, имеющий не~менее $n$ отличных от~нуля
коэффициентов.

\item
Даны многочлены $P(x)$ и~$Q(x)$ десятой степени, старшие коэффициенты которых
равны~$1$.
Известно, что уравнение $P(x) = Q(x)$ не~имеет действительных корней.
Докажите, что уравнение $P(x + 1) = Q(x - 1)$ имеет хотя~бы один действительный
корень.

\item
Дан многочлен нечетной степени $P(x)$.
Докажите, что уравнение  $P(P(x)) = 0$ имеет не~меньше различных действительных
корней, чем уравнение $P(x) = 0$.

\item
Какое наибольшее количество нулевых коэффициентов может быть у~многочлена
степени~$n$, имеющего $n$ различных действительных корней?

\end{problems}

