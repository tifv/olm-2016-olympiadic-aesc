% $date: 2016-02-03 # среда

\section*{Предрегиональный разнобой по геометрии}

% $authors:
% - Глеб Погудин

\begin{problems}

\item
Окружности $\omega_1$ и~$\omega_2$ касаются внешним образом в~точке~$P$.
Через центр $\omega_1$ проведена прямая~$l_1$, касающаяся $\omega_2$.
Аналогично, прямая~$l_2$ касается $\omega_1$ и~проходит через центр $\omega_2$.
Оказалось, что прямые $l_1$ и~$l_2$ непараллельны.
Докажите, что точка~$P$ лежит на~биссектрисе одного из~углов, образованных
$l_1$ и~$l_2$.

\item
Из~вершины~$B$ остроугольного треугольника $ABC$ проведены перпендикуляры
к~сторонам $AB$ и~$BC$ до~пересечения с~прямой~$AC$ в~точках $P$ и~$Q$
соответственно.
Докажите, что описанные окружности треугольников $ABC$ и~$PBQ$ касаются.

\item
Дан выпуклый шестиугольник $ABCDEF$.
Известно, что $\angle FAE = \angle BDC$, а~четырехугольники $ABDF$ и~$ACDE$
являются вписанными.
Докажите, что прямые $BF$ и~$CE$ параллельны.

\item
Докажите, что в~выпуклом четырехугольнике отрезок, соединяющий середины
противоположных сторон, делит диагонали в~одинаковом отношении.

\item
Окружности $S_1$ и~$S_2$ с~центрами $O_1$ и~$O_2$ пересекаются в~точках $A$
и~$B$.
Луч~$O_1 A$ пересекает окружность~$S_2$ в~точке~$M$, луч~$O_2 A$ пересекает
окружность~$S_1$ в~точке~$N$, а~прямая~$MN$ вторично пересекает эти окружности
в~точках $E$ и~$F$.
Докажите, что $AE = AF$.

\item
В~треугольнике $ABC$ угол~$A$ равен $60^{\circ}$.
Докажите, что $IO = IH$, где $I$~--- центр вписанной окружности, $O$~--- центр
описанной окружности, $H$~--- ортоцентр.
% http://www.mit.edu/~evanchen/handouts/Fact5/Fact5.pdf

\item
Выпуклый четырехугольник $ABCD$ таков, что $AB \cdot CD = AD \cdot BC$.
Докажите, что
\[
    \angle BAC + \angle CBD + \angle DCA + \angle ADB
=
    180^{\circ}
\, . \]

\end{problems}

