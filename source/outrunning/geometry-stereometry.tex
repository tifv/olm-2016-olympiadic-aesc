% $date: 2016-01-13 # среда

\section*{Стереометрия}

% $authors:
% - Глеб Погудин

\begin{problems}

\item
Основания трех высот треугольной пирамиды являются точками пересечения медиан
противоположных граней.
Докажите, что все ребра пирамиды равны.
% 11.6 05-06

\item
В~основании четырехугольной пирамиды $SABCD$ лежит параллелограмм $ABCD$.
Докажите, что для любой точки~$O$ внутри пирамиды сумма объемов тетраэдров
$OSAB$ и~$OSCD$ равна сумме объемов тетраэдров $OSBC$ и~$OSDA$.
% 11.6 09-10

\item
Назовем прямую, проходящую через середины скрещивающихся ребер тетраэдра,
\emph{хорошей} средней линией тетраэдра, если она образует равные углы
с~четырьмя прямыми, содержащими остальные ребра тетраэдра.
Докажите, что тетраэдр правильный, если хотя~бы две его средние линии хорошие.
% 11.2 99-00

\item
Точки $A_1$, $B_1$, $C_1$, $D_1$~--- середины ребер $SA$, $SB$, $SC$ и~$SD$
пирамиды $SABCD$.
Известно, что отрезки $AC_1$, $BD_1$, $CA_1$ и~$DB_1$ проходят через одну точку
и~имеют равные длины.
Докажите, что $ABCD$~--- прямоугольник.
% 11.6 96-97

\item
Пусть $A_1$, $B_1$, $C_1$, $D_1$~--- соответственно середины ребер $SA$, $SB$,
$SC$, $SD$ четырехугольной пирамиды $SABCD$.
Известно, что пространственные четырехугольники $A B C_1 D_1$, $A_1 B C D_1$,
$A_1 B_1 C D$, $A B_1 C_1 D$ являются плоскими и~имеют равные площади.
Докажите, что $ABCD$~--- ромб.
% 11.7 02-03

\item
Пятигранник $A B C A_1 B_1 C_1$ имеет две непараллельные треугольные
грани $ABC$ и~$A_1 B_1 C_1$ и~три грани~--- выпуклые четырехугольники
$A B B_1 A_1$, $B C C_1 B_1$, $C A A_1 C_1$.
Докажите, что плоскость, проведенная через точки пересечения диагоналей
четырехугольных граней, содержит прямую пересечения плоскостей $ABC$
и~$A_1 B_1 C_1$.
% 11.4 04-05

\item
Пусть $ABCD$~--- тетраэдр, $\omega$~--- сфера, касающаяся всех его ребер.
Две точки касания сферы~$\omega$ с~ребрами тетраэдра $ABCD$ соединим отрезком
тогда и~только тогда, когда они лежат на~одной грани тетраэдра.
Докажите, что сумма всех таких отрезков меньше, чем $3 (AI + BI + CI + DI)$,
где $I$~--- центр сферы~$\omega$.
% 11.4 01-02

\end{problems}

