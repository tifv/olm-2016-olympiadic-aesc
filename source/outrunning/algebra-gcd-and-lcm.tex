% $date: 2015-11-18 # среда

\section*{НОДы и НОКи}

% $authors:
% - Глеб Погудин

\begingroup
    \def\abs#1{\lvert #1 \rvert}
    \def\GCD{\operatorname{\text{НОД}}}
    \def\LCM{\operatorname{\text{НОК}}}

\begin{problems}

\item
Для натуральных $a$ и~$b$ выполнено
$\GCD(a, b) + \LCM(a, b) = a + b$.
Докажите, что из~этих двух чисел одно делится на~другое.

\item
О~натуральных числах $a$, $p$, $q$ известно, что $a p + 1$ делится на~$q$,
а~$a q + 1$ делится на~$p$.
Докажите, что
\(
    a > p q / \bigl(2 (p + q) \bigr)
\).

\item
Найдите все натуральные $a$ и~$b$ такие, что
\[
    \LCM(a, b) - \GCD(a, b)
=
    \frac{a b}{5}
\; . \]

\item
Натуральные числа $a$ и~$b$ таковы, что
$a \cdot \GCD(a, b) + b \cdot \LCM(a, b) < 2{,}5 a b$.
Докажите, что $a$ делится на~$b$.

\item
Даны натуральные числа $a$ и~$b$ такие, что $(a + 1) / b + (b + 1) / a$
является целым.
Докажите, что наибольший общий делитель чисел $a$ и~$b$ не~превосходит числа
$\sqrt{a + b}$.

\item
$a$ и~$b$~--- различные натуральные числа такие, что, $a b (a + b)$ делится
на~$a^2 + a b + b^2$.
Докажите, что $\abs{a - b} > \sqrt[3]{a b}$.

\item
Докажите, что если $\LCM(a, a + 5) = \LCM(b, b + 5)$, то~$a = b$.

\item
Могут~ли $\LCM(a, b)$ и~$\LCM(a + c, b + c)$ быть равны?

\item
Пусть $m$ и~$n$ взаимно просты.
Выразите $\GCD(5^{n} + 7^{n}, 5^{m} + 7^{m})$ через $m$ и~$n$.

\item
Для любых натуральных чисел $m$ и~$n$ ($n > m$) докажите неравенство
\[
    \LCM(m, n) + \LCM(m + 1, n + 1) > 2 m n / \sqrt{n - m}
\, . \]

\end{problems}

\endgroup % \def\GCD \def\LCM

