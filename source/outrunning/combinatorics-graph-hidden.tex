% $date: 2016-01-22 # пятница

\section*{Графы. Кто ищет~--- тот всегда найдёт!}

% $authors:
% - Антон Гусев

\begin{problems}

\item
Куб $n \times n \times n$ разбит на~кубики $1 \times 1 \times 1$.
Какое минимальное количество граней $1 \times 1$ необходимо в~нем убрать, чтобы
из~любой его части можно было пробраться наружу?

\item
Из~$54$ одинаковых единичных картонных квадратов сделали незамкнутую цепочку,
соединив их шарнирно вершинами.
Любой квадрат (кроме крайних) соединен с~соседями двумя противоположными
вершинами.
Можно~ли этой цепочкой квадратов закрыть поверхность куба
$3 \times 3 \times 3$?

\item
На~плоскости нарисовано $n$~кругов, причем любые два круга не~пересекаются,
но~могут касаться.
Каково максимальное количество точек касания?

\item
Из~клетчатой доски, раскрашенной шахматным образом, вырезана связная
(по~сторонам клеток) фигура, содержащая $n$ черных клеток.
\\
\subproblem
Сколько максимум у~нее может быть белых клеток?
\\
\subproblem
У~фигуры ровно $3n$ белых клеток.
Докажите, что ее можно разрезать на~четырехклеточные буквы <<Т>>.

\item
$N^3$ единичных кубиков просверлены по~диагонали и~плотно нанизаны на~нить,
после чего нить связана в~кольцо (т.~е. вершина первого кубика соединена
с~вершиной последнего).
При каких $N$ из~получившегося <<ожерелья>> можно сложить куб с~ребром~$N$?

\item
Дано натуральное число $n > 2$.
Рассмотрим все покраски клеток доски $n \times n$ в~$k$~цветов такие, что
каждая клетка покрашена ровно в~один цвет, и~все $k$~цветов встречаются.
При каком наименьшем $k$ в~любой такой покраске найдутся четыре окрашенных
в~четыре разных цвета клетки, расположенные в~пересечении двух строк и~двух
столбцов?

\item
Петя поставил на~доску $50 \times 50$ несколько фишек, в~каждую клетку~---
не~больше одной.
Докажите, что Вася может поставить на~свободные поля этой~же доски не~более
$99$ новых фишек (возможно, ни~одной) так, чтобы по-прежнему в~каждой клетке
стояло не~больше одной фишки, и~в~каждой строке и~каждом столбце этой доски
оказалось четное количество фишек.

\end{problems}

