% $date: 2015-10-02 # пятница

\section*{Разнобой}

% $authors:
% - Андрей Меньщиков

\begin{problems}

\item
Докажите, что доску $12 \times 12$ можно разрезать на~прямоугольники
$1 \times 2$ более, чем $10^{14}$ способами.

\item
Выпуклый четырехугольник $ABCD$ таков, что $AC + BD = 20$, $AB + CD = 12$.
Какое наибольшее значение может принимать площадь четырехугольника $ABCD$?

\item
Дано натуральное числоЁ$n$ и~набор различных натуральных чисел
$a_1, a_2, \ldots, a_{2n}$, каждое из~которых не~превосходит $n^2$.
Докажите, что в~множестве попарных разностей этого набора какое-то ненулевое
число встретится не~менее трех раз.

\item
Существует~ли такая бесконечная последовательность натуральных чисел
$a_1, a_2, \ldots$, что для любого натурального~$n$ выполнено соотношение
\(
    \frac{1}{a_n}
=
    \frac{1}{a_{n+1}} + \frac{1}{a_{n+2}}
\)?

\item
В~равнобедренном треугольнике $ABC$ ($AB = AC$) провели такую чевиану~$AX$, что
радиус вписанной в~треугольник $ABX$ окружности равен радиусу вневписанной
в~треугольник $ACX$ окружности, касающейся отрезка~$CX$.
Докажите, эти радиусы равны четверти длины высоты треугольника $ABC$, опущенной
на~боковую сторону.

\item
Найдите максимальное количество ребер в~графе на~$n \geq 6$ вершинах, где любые
два цикла имеют общую вершину.

\end{problems}

