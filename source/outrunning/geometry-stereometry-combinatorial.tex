% $date: 2016-01-15 # пятница

\section*{Комбинаторная стереометрия}

% $authors:
% - Глеб Погудин

\begin{problems}

\item
Существует~ли треугольная пирамида, каждое ребро основания которой видно
из~середины противолежащего бокового ребра под прямым углом?
% 11.2 95-96

\item
Боковое ребро четырехугольной пирамиды назовем \emph{хорошим,} если медианы
двух содержащих его граней, проведенные в~середину этого ребра, равны.
Докажите, что если в~пирамиде три боковых ребра~--- хорошие, то~четвертое
боковое ребро также является хорошим.
% 11.3 06-07

\item
Назовем \emph{кубоподобным} многогранник, имеющий шесть граней и~восемь вершин,
в~каждой из~которых сходятся по~три грани, каждая грань при этом~---
четырехугольник.
Докажите, что если отрезки, соединяющие точки пересечения диагоналей
противоположных граней кубоподобного многогранника пересекаются в~одной точке,
то~отрезки, соединяющие противоположные вершины (главные диагонали), также
пересекаются в~одной точке.
% 11.7 98-99

\item
В~выпуклом многоугольнике~$P_1$ содержится выпуклый многоугольник~$P_2$.
Докажите, что при любой гомотетии относительно точки $x \in P_2$
с~коэффициентом $k = -1 / 2$ по~крайней мере одна вершина~$P_2$ не~выйдет
за~пределы $P_1$.
% 11.7 00-01

\item
Найдите геометрическое место точек~$P$, лежащих внутри куба $ABCDA'B'C'D'$, для
которых в~каждую из~шести пирамид $PABCD$, $PABB'A'$, $PBCC'B'$, $PCDD'C'$,
$PDAA'D'$, $PA'B'CD'$ можно вписать в~сферу.
% 11.7 03-04

\item
Каждую грань тетраэдра можно поместить в~круг радиуса~$1$.
Докажите, что весь тетраэдр можно поместить в~шар радиуса
$3 \sqrt{2} / 4$.
% 11.4 07-08

\item
В~пространстве расположены четыре попарно скрещивающиеся прямые.
Докажите, что найдется полуплоскость, границей которой является одна из~этих
прямых, не~пересекающаяся с~остальными тремя прямыми.
% 11.8 97-98

\end{problems}

