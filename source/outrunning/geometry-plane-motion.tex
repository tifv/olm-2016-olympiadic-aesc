% $date: 2015-09-25 # пятница

\section*{Движения плоскости}

% $authors:
% - Андрей Меньщиков

\begin{problems}

\item
Равные окружности $S_1$ и~$S_2$ касаются окружности~$S$ внутренним образом
в~точках $A_1$ и~$A_2$.
Произвольная точка~$C$ окружности~$S$ соединена отрезками с~точками $A_1$ и~$A_2$.
Эти отрезки пересекают окружности $S_1$ и~$S_2$ в~точках $B_1$ и~$B_2$.
Докажите, что $A_1 A_2 \parallel B_1 B_2$.

\item
На~отрезке~$AE$ по~одну сторону от~него построены равносторонние треугольники
$ABC$ и~$CDE$ (точка~$C$ лежит на~отрезке~$AE$).
Точки $M$ и~$P$~--- середины отрезков $AD$ и~$BE$.
Докажите, что треугольник $CPM$~--- равносторонний.

\item\claim{Лемма Архимеда}
Пусть $A$ и~$B$~--- фиксированные точки окружности~$S$.
Выберем одну из~дуг окружности~$S$ с~концами $A$ и~$B$ и~рассмотрим
произвольную окружность, касающуюся отрезка~$AB$ и~выбранной дуги.
Обозначим точки касания через $P$ и~$Q$ соответственно.
Докажите, что все прямые~$PQ$ пересекаются в~одной точке.

\item\claim{Точка Торричелли}
Пусть $T$~--- такая точка плоскости, что сумма расстояний от~нее до~вершин
данного остроугольного треугольника минимальна.
Докажите, что все стороны треугольника видны из~нее под углом $120^{\circ}$.

\item
Окружность пересекает стороны $AC$, $BC$ и~$AB$ положительно ориентированного
треугольника $ABC$ в~точках $B_2$ и~$B_1$, $A_2$ и~$A_1$, $C_1$ и~$C_2$
(в~порядке обхода по~часовой стрелке).
Оказалось, что перпендикуляры к~сторонам $AC$, $BC$ и~$AB$, восстановленные
в~точках $B_2$, $A_1$ и~$C_1$ соответственно, пересекаются в~одной точке.
Докажите, что перпендикуляры к~тем~же сторонам, восстановленные в~точках
$B_1$, $A_2$ и~$C_2$, также пересекаются в~одной точке.

\item
В~квадрате со~стороной~$1$ расположена фигура, расстояние между любыми двумя
точками которой не~равно $0{,}001$.
Докажите, что площадь этой фигуры не~превосходит
\\
\subproblem $0{,}34$;
\quad
\subproblem $0{,}287$.

\end{problems}

