% $date: 2015-09-30 # среда

\section*{Графы. Добавка про деревья}

% $authors:
% - Глеб Погудин

\begin{problems}

\item
Доказать, что в~дереве пересечение поддеревьев является поддеревом.

\item
Докажите, что вершины любого дерева можно покрасить в~два цвета таким образом,
чтобы смежные вершины имели разные цвета.

\item
Есть $n$~коробок, к~каждой лежит по~одному утюгу, все утюги~--- разные.
У~вас есть кнопки, на~которых написано
<<поменять местами содержимое $i$-ой и~$j$-ой коробки>>.
\\
\subproblem
Докажите, что можно выбрать такие $(n - 1)$ кнопок, чтобы используя только их
можно было~бы реализовать любую перестановку утюгов;
\\
\subproblem
Докажите, что меньшим числом кнопок обойтись нельзя.

\item
В~группе людей каждый имеет знакомого.
Докажите, что эту группу можно разбить на~две так, чтобы каждый человек имел
знакомого из~другой группы.

\item
В~связном графе есть остовное дерево с~$m$ висячими вершинами и~с~$n$ висячими
вершинами.
Докажите, что для любого $k$ такого, что $m < k < n$, в~этом графе найдется
остовное дерево с~$k$ висячими вершинами.

\end{problems}

