% $date: 2016-03-02 # среда

\section*{Комбигеом}

% $authors:
% - Глеб Погудин

\begin{problems}

\item
На~плоскости дано~$300$ точек, никакие~$3$ из~которых не~лежат на~одной прямой.
Докажите, что существует~$100$ попарно не~пересекающихся треугольников
с~вершинами в~этих точках.

\item
Шарообразная планета окружена $25$ точечными астероидами.
Доказать, что в~любой момент на~поверхности планеты найдется точка, из~которой
астроном не~сможет наблюдать более $11$~астероидов.

\item
В~круге радиуса~$16$ расположено $650$ точек.
Докажите, что найдется кольцо с~внутренним радиусом~$2$ и~внешним радиусом~$3$,
в~котором лежит не~менее $10$ из~данных точек.

\item
Несколько точек на~плоскости расположены так, что любой треугольник с~вершинами
в~этих точках имеет площадь не~больше $1$.
Доказать, что все эти точки можно поместить в~треугольник площади~$4$.

\item
На~прямоугольном листе бумаги отмечены
\\
\subproblem несколько точек на~одной прямой;
\qquad
\subproblem три точки.
\\
Разрешается сложить лист бумаги несколько раз по~прямой так, чтобы отмеченные
точки не~попали на~линии сгиба, и~затем один раз шилом проколоть сложенный лист
насквозь.
Докажите, что это можно сделать так, чтобы дырки оказались в~точности
в~отмеченных точках и~лишних дырок не~получилось.

\item
Найдите все конечные множества точек на~плоскости, обладающие тем свойством,
что никакие три точки множества не~лежат на~одной прямой и~вместе с~каждыми
тремя точками данного множества точка пересечения высот треугольника,
образованного этими точками, также принадлежит данному множеству.

\item
На~плоскости даны $n$~точек, никакие три из~которых не~лежат на~одной прямой.
Сколькими различными способами это множество точек можно разбить на~два
непустых подмножества так, чтобы выпуклые оболочки этих подмножеств
не~пересекались?

\item
Имеется набор векторов $v_1, \ldots, v_n$ на~плоскости такой, что
$v_1 + \ldots + v_n = 0$.
Докажите, что найдется выпуклый многоугольник $A_1 \ldots A_n$ такой, что
$\overline{A_i A_{i + 1}} = v_i$.

\end{problems}

