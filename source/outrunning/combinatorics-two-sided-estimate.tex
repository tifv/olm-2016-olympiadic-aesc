% $date: 2016-01-27 # пятница

\section*{Оценка + пример}

\begin{problems}

\item
Число называется \emph{несложным}, если оно является произведением ровно двух
простых (быть может, равных).
Какое наибольшее количество несложных чисел может идти подряд?

\item
На~новом сайте зарегистрировалось $2000$ человек.
Каждый пригласил к~себе в~друзья по~$1000$ человек.
Два человека объявляются друзьями тогда и~только тогда, когда каждый из~них
пригласил другого в~друзья.
Какое наименьшее количество пар друзей могло образоваться?

\item
В~наборе несколько гирь, все веса которых различны.
Известно, что если положить любую пару гирь на~левую чашу, можно весы
уравновесить, положив на~правую чашу одну или несколько гирь из~остальных.
Найдите наименьшее возможное число гирь в~наборе.

\item
Петя красит клетки таблицы $n \times n$ по~следующему правилу: если какая-то
незакрашенная клетка граничит по~стороне с~двумя закрашенными, то~ее можно
закрасить.
Какое наименьшее число клеток могло быть закрашено изначально, если известно,
что Петя смог закрасить все клетки?

\item
Два муравья проползли каждый по~своему замкнутому маршруту
на~доске $7 \times 7$.
Каждый полз только по~сторонам клеток доски и~побывал в~каждой из~$64$ вершин
клеток ровно один раз.
Каково наименьшее возможное число таких сторон, по~которым проползали и~первый,
и~второй муравьи?

\item
Расстоянием между числами $\overline{a_1 a_2 a_3 a_4 a_5}$
и~$\overline{b_1 b_2 b_3 b_4 b_5}$ назовем
максимальное $i$, для которого $a_i \neq b_i$.
Все пятизначные числа выписаны друг за~другом в~некотором порядке.
Какова при этом минимально возможная сумма расстояний между соседними числами?

\iffalse
\item
На~окружности расположена тысяча непересекающихся дуг, и~на~каждой из~них
написаны два натуральных числа.
Сумма чисел каждой дуги делится на~произведение чисел дуги, следующей за~ней
по~часовой стрелке.
Каково наибольшее возможное значение наибольшего из~написанных чисел?
\fi

\item
В~какое наибольшее число цветов можно раскрасить все клетки доски размера
$10 \times 10$ так, чтобы в~каждой строке и~в~каждом столбце находились клетки
не~более, чем пяти различных цветов?

\item
Какое минимальное количество клеток можно закрасить черным в~белом квадрате
$300 \times 300$, чтобы никакие три черные клетки не~образовывали уголок,
а~после закрашивания любой белой клетки это условие нарушалось?

\item
За~круглым столом сидит компания из~тридцати человек.
Каждый из~них либо дурак, либо умный.
Всех сидящих спрашивают: <<Кто Ваш сосед справа -- умный или дурак?>>.
В~ответ умный говорит правду, а~дурак может сказать как правду, так и~ложь.
Известно, что количество дураков не~превосходит $F$.
При каком наибольшем значении $F$ всегда можно, зная эти ответы, указать
на~умного человека в~этой компании?

\end{problems}

