% $date: 2015-11-11 # среда

\section*{Инвариант}

% $authors:
% - Антон Гусев

\begin{problems}

\item
На~доске написано число 1234.
Разрешается к~числу либо прибавить по~единице к~двум соседним цифрам, либо
вычесть по~единице (если среди них нет девяток, нулей соответственно).
Можно~ли получить число 2013?

\item
На~доске написаны числа $1, 2, 3, \ldots, 20$.
Разрешается стереть любые два числа $a$ и~$b$ и~записать вместо них
число $ab + a + b$.
Какое число может остаться на~доске после 19 таких операций?

\item
Какое максимальное количество диагональных ходов может совершить король, обойдя
всю шахматную доску по~пути без самопересечений?
(Король ходит на~любую соседнюю клетку по~стороне или вершине)

\item
Муравей ползает по~замкнутому маршруту по~ребрам додэкаэдра, нигде
не~разворачиваясь назад.
Маршрут проходит ровно по~два раза по~каждому ребру.
Докажите, что некоторое ребро муравей проходит оба раза в~одном и~том~же
направлении.

\end{problems}

