% $date: 2015-09-23 # среда

\section*{Графы. Связность}

% $authors:
% - Глеб Погудин

\begin{problems}

\item
В~графе на~$n$ вершинах степень каждой не~меньше $n / 2$.
Докажите, что этот граф связен.

\item
В~группе из~нескольких человек некоторые люди знакомы друг с~другом,
а~некоторые~--- нет.
Каждый вечер один из~них устраивает ужин для всех своих знакомых и~знакомит их
друг с~другом.
После того как каждый человек устроил хотя~бы один ужин, оказалось, что
какие-то два человека все еще не~знакомы.
Докажите, что на~следующем ужине им познакомиться тоже не~удастся.

\item
В~стране $100$ городов, некоторые из~которых соединены авиалиниями.
Известно, что от~любого города можно долететь до~любого другого
(возможно, с~пересадками).
Докажите, что можно побывать в~каждом городе, совершив не~более:
\\
\subproblem 198 перелетов;
\quad
\subproblem 196 перелетов.

\item
В~связном графе на~$100$ вершинах $199$ ребер.
Докажите, что можно удалить все ребра некоторого цикла так, чтобы граф остался
связен.

\item
В~стране $n$~городов, некоторые пары из~которых соединены непересекающимися
дорогами.
Известно, что из~любого города можно добраться по~дорогам до~любого другого,
причем единственным способом (если не~проезжать по~одной дороге более одного
раза).
Докажите, что министр может объявить не~более, чем $n / 51$ городов
закрытыми (и~запретить въезд в~них и~выезд из~них) так, чтобы после этого для
любой пары городов $X$, $Y$ выполнялось одно из~двух условий:
либо из~$X$ нельзя добраться до~$Y$,
либо из~$X$ можно добраться до~$Y$, проехав не~более, чем по~$49$~дорогам.

\item
Известно, что в~графе~$G$ из~вершины~$u$ в~вершину~$v$ можно добраться при
удалении любой другой вершины $G$.
Докажите, что между $u$ и~$v$ есть два непересекающихся пути.

\item
Хозяйка испекла для гостей пирог.
За~столом может оказаться либо $p$, либо $q$~человек.
На~какое минимальное количество кусков (не~обязательно равных) нужно заранее
разрезать пирог, чтобы в~любом случае его можно было раздать поровну между
гостями, если:
\\
\subproblem $p$ и~$q$ взаимно просты;
\\
\subproblem $p$ и~$q$ имеют наибольший общий делитель $d$?

%\item
%В~стране $100$ городов, из~каждого города выходит хотя~бы одна дорога.
%Докажите, что можно закрыть несколько дорог так, чтобы по-прежнему из~каждого
%города выходило не~менее одной дороги и~при этом по~крайней мере
%из~$67$~городов выходило ровно по~одной дороге.

\end{problems}

