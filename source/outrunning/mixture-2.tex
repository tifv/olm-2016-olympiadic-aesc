% $date: 2015-12-11 # пятница

\section*{Разнобой}

% $authors:
% - Андрей Меньщиков

\begin{problems}

\item
На~столе лежат картинками вниз $8$ игральных карт.
Вы можете указать на~любую группу карт (в~частности, на~одну карту, или,
например, на~все $8$) и~спросить, сколько карт бубновой масти в~этой группе.
В~качестве ответа вам сообщат число, отличающееся от~истинного значения на~$1$.
Можно~ли при помощи $5$ вопросов узнать число бубновых карт, лежащих на~столе?

\item
Ожерелье состоит из~$100$ синих и~некоторого количества красных бусин,
нанизанных на~нить в~форме окружности.
Известно, что на~любом отрезке ожерелья, содержащем $8$~бусин, есть не~менее
$5$ красных.
Какое наименьшее количество красных бусин может быть в~ожерелье?

\item
$P(x)$~--- квадратный трехчлен.
Какое наибольшее количество членов, равных сумме двух предыдущих членов, может
быть в~последовательности
%\(
%    P(1), P(2), P(3), \ldots
%\)?
\[
    P(1), P(2), P(3), \ldots
\, ? \]

\item
В~клетках таблицы $10 \times 100$ написаны целые числа.
Разрешается выбрать любую клетку и~вычесть из~стоящего в~ней числа количество
соседних по~стороне клеток, а~к~числам, стоящим в~соседних клетках, прибавить
по~$1$.
Всегда~ли из~таблицы, сумма чисел в~которой равна $0$, такими операциями можно
получить таблицу, целиком заполненную нулями?

\item
На~высотах треугольника $ABC$ отложили отрезки $A A_1$, $B B_1$, $C C_1$,
равные диаметру вписанной окружности (высота~--- это отрезок).
Эта окружность касается сторон $BC$, $AC$, $AB$ в~точках $A'$, $B'$, $C'$.
Докажите, что прямые $A_1 A'$, $B_1 B'$, $C_1 C'$ пересекаются в~одной точке.

\item
Дано натуральное число~$n$.
Рассмотрим все наборы целых неотрицательных чисел
\[
    x_1, x_2, \ldots, x_n
\quad\text{такие, что\enspace$x_{1} + x_{2} + \ldots + x_{n} = n - 1$.}
\]
Найдите наибольший общий делитель всех произведений вида
\[
    C_{n}^{x_{1}} \cdot
    C_{n}^{x_{2}} \cdot
    \ldots \cdot
    C_{n}^{x_{n}}
\, . \]

\end{problems}

