% $date: 2015-09-16 # среда

\section*{Многочлены с целыми коэффициентами}

% $authors:
% - Глеб Погудин

% $matter[full-version,no-header]:
% - .[-full-version]
% - ../algebra-polynomial-integer-more

\begin{problems}

\item
На~графике многочлена с~целыми коэффициентами отмечены две точки с~целыми
координатами.
Докажите, что если расстояние между ними~--- целое число, то~соединяющий их
отрезок параллелен оси абсцисс.

\item
\subproblem
Докажите, что если многочлен $f(x) = a_n x^n + \ldots + a_1 x + a_0$
с~целыми коэффициентами принимает при пяти целых значениях~$x$ значение~$7$,
то~он не~может принимать значение~$14$ ни~при каком целом значении~$x$.
\\
\subproblem
То~же самое при четырех значениях.

\item
Приведенный квадратный трехчлен с~целыми коэффициентами в~трех последовательных
целых точках принимает простые значения.
Докажите, что он принимает простое значение по~крайней мере еще в~одной целой
точке.

\item
Докажите, что для любого многочлена~$P$ с~целыми коэффициентами и~любого
натурального~$k$ существует такое натуральное~$n$, что $P(1) + \ldots + P(n)$
делится на~$k$.

\item
\subproblem
Докажите, что не~существует многочлена (степени больше нуля) с~целыми
коэффициентами, принимающего при каждом натуральном значении аргумента
значение, равное некоторому простому числу.
\\
\subproblem
Докажите, что не~существует многочлена степени не~ниже двух
с~целыми неотрицательными коэффициентами, значение которого при любом
простом~$p$ является простым числом.
\\
\subproblem
Докажите, что не~существует многочлена (степени больше нуля)
с~целыми коэффициентами, для которого множество простых делителей ненулевых
значений в~целых точках конечно.

\item
Дано $n$~чисел, $p$~--- их произведение.
Разность между $p$ и~каждым из~этих чисел~--- нечетное число.
Докажите, что все данные $n$~чисел иррациональны.

\end{problems}

