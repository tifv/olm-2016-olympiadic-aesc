% $date: 2015-10-21 # среда

\section*{Теорема Шаля}

% $authors:
% - Глеб Погудин

\begin{problems}

\item
Чему равна композиция
\\
\subproblem двух поворотов?
\quad
\subproblem двух симметрий?
\\
\subproblem симметрии и поворота?
\\
\subproblem симметрии и параллельного переноса?
\\
\subproblem поворота и параллельного переноса?
\\
\subproblem двух центральных симметрий?

\item
Докажите, что любое движение однозначно задается образами трех неколлинеарных
точек.

\item
Докажите, что всякое собственное движение плоскости является или поворотом, или
параллельным переносом.

\item
Докажите, что всякое несобственное движение плоскости является скользящей
симметрией.

\item
На плоскости имеется многоугольник с нечетным числом сторон.
Точку~$M$ последовательно отражают относительно середин сторон многоугольника,
получается точка~$N$.
Докажите, что середина $MN$ является вершиной многоугольника.

\item
Докажите, что любое движение является композицией не более, чем трех симметрий.

\item
Дан треугольник $ABC$.
Постройте точки $K$, $L$ и $M$ такие, чтобы треугольники $AKM$, $BLK$ и $CLM$
были правильными.

\item
На плоскости нарисованы два равных треугольника $A_1 B_1 C_1$ и $A_2 B_2 C_2$,
причем направление обхода вершин у них разное.
Докажите, что середины отрезков $A_1 A_2$, $B_1 B_2$ и $C_1 C_2$ лежат на одной
прямой.

\item
Стороны выпуклого четырехугольника проходят через вершины параллелограмма.
Известно, что три из них делятся этими вершинами пополам.
Докажите, что четвертая тоже делится пополам.

%\item
%С помощью циркуля и линейки восстановить пятиугольник по серединам его сторон.

\item
Дан треугольник $ABC$.
Произвольная точка плоскости $M$ последовательно отражается относительно прямых
$AB$, $AC$ и $BC$.
Результат этого действа обозначим через $T(M)$.
Найдите множество точек~$M$ таких, что расстояние между $M$ и $T(M)$
минимально.

\end{problems}

