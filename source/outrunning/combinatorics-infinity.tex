% $date: 2016-03-16 # среда

\section*{Бесконечность}

% $authors:
% - Глеб Погудин

\begin{problems}

\item
Дано бесконечное число углов.
Докажите, что этими углами можно покрыть плоскость.

\item
В~стране Фибоначчи есть купюры достоинством
$1$, $2$, $3$, $5$, $8$, $13$, $21$, $34$, $55$ лир.
У~Леонардо есть купюра $55$~лир.
Каждый день он может пойти в~банк и~обменять любую имеющуюся у~него купюру
на~любое количество купюр меньшего достоинства.
Кроме того, каждый день Леонардо должен тратить $1$~лиру на~еду.
Докажите, что Леонардо сможет существовать сколь угодно долго, но~не~бесконечно
долго.

\item
Есть бесконечное дерево, подвешенное за~корень.
Степень каждой вершины конечна.
Докажите, что существует бесконечный путь, идущий от~корня вниз.

\item
Докажите, что из~бесконечной последовательности различных натуральных чисел
можно выделить бесконечную возрастающую подпоследовательность.

\item
Натуральные числа раскрасили в~два цвета.
Обязательно~ли существует одноцветная бесконечная арифметическая прогрессия?

\end{problems}

