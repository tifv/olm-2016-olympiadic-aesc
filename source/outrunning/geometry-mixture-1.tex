% $date: 2015-12-02 # среда

\section*{Геометрический разнобой}

% $authors:
% - Глеб Погудин
% Глеб Погудин скоммуниздил у Ванечки с Владимиром

\begin{problems}

\item
Окружность, вписанная в~прямоугольный треугольник $ABC$ с~гипотенузой~$AB$,
касается его сторон $BC$, $AC$ и~$AB$ в~точках $A_1$, $B_1$ и~$C_1$
соответственно.
$B_1 H$~--- высота треугольника $A_1 B_1 C_1$.
Докажите, что $H$ лежит на~биссектрисе угла $CAB$.

\item
В~треугольнике $ABC$ проведена биссектриса~$A A_1$.
На~стороне~$AB$ выбрана точка~$K$ так, чтобы $B K = B A_1$.
Биссектриса угла~$C$ пересекает $A_1 K$ в~точке~$P$.
Докажите, что $P A = P A_1$.

\item
Окружности $\omega_1$ и~$\omega_2$ пересекаются в~точках $A$ и~$B$.
Касательная к~$\omega_1$ в~точке~$B$ пересекает второй раз $\omega_2$
в~точке~$P$.
Касательная к~$\omega_2$ в~точке~$B$ пересекает второй раз $\omega_1$
в~точке~$Q$.
Прямая~$QA$ второй раз пересекает $\omega_2$ в~точке~$R$.
Докажите, что $BR = BP$.

\item
Пусть $A$, $B$ и~$C$ лежат на~окружности, а~прямая~$b$ касается этой окружности
в~точке~$B$.
Из~точки $P$, лежащей на~прямой~$b$, опущены перпендикуляры $P A_1$ и~$P C_1$
на~прямые $AB$ и~$BC$ соответственно (точки $A_1$ и~$C_1$ лежат на~отрезках
$AB$ и~$BC$).
Докажите, что $AC \perp A_1 C_1$.

\item
На~сторонах треугольника $ABC$ во~внешнюю границу, как на~основаниях, построены
равнобедренные треугольники $BCD$, $CAE$ и~$ABF$.
Докажите, что прямые, проходящие через точки $A$, $B$ и~$C$ перпендикулярно
$EF$, $FD$ и~$DE$ соответственно, пересекаются в~одной точке.

\item
Пусть в~четырехугольнике $ABCD$ точка $M$~--- это середина стороны~$AD$,
$\angle BMC = 90^{\circ}$, $\angle BAD = \angle BCM$.
Докажите, что тогда прямые $AB$ и~$CD$ пересекаются под прямым углом.

\item
На~плоскости отмечены два отрезка.
Найдите ГМТ точек плоскости, из~которых эти отрезки видны под равными углами.

\end{problems}

