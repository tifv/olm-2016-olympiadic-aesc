% $date: 2015-11-11 # среда

\section*{Полуинвариант}

% $authors:
% - Антон Гусев

\begin{problems}

\item
Таблица $15 \times 15$ заполнена плюсами и~минусами.
Разрешается выбрать любую строку или любой столбец и~поменять все стоящие там
знаки на~противоположные.
Докажите, что несколькими такими операциями можно добиться того, чтобы в~каждой
строке и~в~каждом столбце плюсов было больше, чем минусов.

\item
На~плоскости дано $2 N$~точек, никакие три из~которых не~лежат на~одной прямой,
$N$ из~них окрашены в~красный цвет, остальные в~синий.
Докажите, что эти точки можно соединить $N$ непересекающимися отрезками, каждый
из~которых будет соединять красную точку с~синей.

\item
По~окружности расставлены $n$~чисел.
Если подряд стоят числа $a$, $b$, $c$ и~$d$ и~при этом $(a - d) (b - c) > 0$,
то~числа $b$ и~$c$ разрешается поменять местами.
Докажите, что через несколько шагов нам не~удастся произвести ни~одной такой
перестановки.

\item
На~окружности сидят 12~кузнечиков в~различных точках.
Эти точки делят окружность на~12 дуг.
Отметим 12~середин этих дуг.
По~сигналу кузнечики одновременно прыгают, каждый~--- в~ближайшую по~часовой
стрелке отмеченную точку.
Снова образуются 12~дуг, прыжки в~середины дуг повторяются и~т.~д.
Может~ли хотя~бы один кузнечик вернуться в~свою исходную точку после того, как
им сделано
\\
\subproblem 12~прыжков?
\qquad
\subproblem 13~прыжков?

\item
Шахматная доска разбита на~доминошки.
К~правой верхней клетке добавлена одна клетка справа
(первая строка состоит из~9 клеток, остальные — из~8).
Разрешается вынимать любую доминошку и~класть ее на~две пустые соседние клетки.
Докажите, что все доминошки можно расположить горизонтально.

\end{problems}

