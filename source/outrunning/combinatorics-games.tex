% $date: 2015-12-16 # среда

\section*{Игры}

% $authors:
% - Погудин Глеб

\begin{problems}

\item
Петя и~Вася играют на~доске размером $7 \times 7$.
Они по~очереди ставят в~клетки доски цифры от~$1$ до~$7$ так, чтобы ни~в~одной
строке и~ни~в~одном столбце не~оказалось одинаковых цифр.
Первым ходит Петя.
Проигрывает тот, кто не~сможет сделать ход.
Кто из~них сможет выиграть, как~бы ни~играл соперник?

\item
Двое поочередно пишут на~доске $5 \times 5$ что-то.
Первый~--- крестики, второй~--- нолики.
Первый выиграет, если напишет пять крестиков в~ряд, второй~--- если ему
помешает.
Кто выигрывает при правильной игре?

\item
Двое играют в~крестики-нолики на~бесконечной клетчатой бумаге по~таким
правилам: первый ставит два крестика, второй~--- нолик, первый~--- снова два
крестика, второй~--- нолик и~т.~д.
Первый выигрывает, когда на~одной вертикали или горизонтали стоит рядом
$k$~крестиков.
Докажите, что первый всегда может добиться победы.
\\
\subproblem $k = 6$.
\qquad
\subproblem $k = 100$.

\item
Двое поочередно пишут на~доске $5 \times 5$ что-то.
Первый~--- единички, второй~--- нули.
Когда доска заполнится, подсчитывают сумму чисел в~каждом
квадрате $3 \times 3$, максимум из этих чисел объявляется выигрышем первого.
Какой максимальный выигрыш может он себе обеспечить?

\item
Двое поочередно пишут $S$ или $O$ в~полоску $1 \times 2000$.
Выигрывает тот, кто первым напишет $SOS$.
Докажите, что второй выиграет при правильной игре.

\item
Двое поочередно располагают доминошки на~доске $m \times n$, где $m n$ четно.
Первый кладет доминошки всегда вертикально, второй~--- горизонтально.
Проигрывает тот, кто не~может сделать ход.
Для всяких $m$ и~$n$ определите, кто выигрывает при правильной игре?

\end{problems}

