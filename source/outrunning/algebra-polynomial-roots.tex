% $date: 2015-10-07 # среда

\section*{Корней много не бывает!}

% $authors:
% - Глеб Погудин

\begin{problems}

\item
На~плоскости расположено $100$ точек.
Известно, что через каждые четыре из~них проходит график некоторого квадратного
трехчлена.
Докажите, что все $100$ точек лежат на~графике одного квадратного трехчлена.

\item
Опишите многочлены $f(x)$ степени не~выше~$3$, которые удовлетворяют условиям:
$f(0) = 1$, $f(1) = 3$, $f(2) = 3$.

\item
Гриша записал на~доске $100$ чисел.
Затем он увеличил каждое число на~$1$ и~заметил, что произведение всех $100$
чисел не~изменилось.
Он опять увеличил каждое число на~$1$, и~снова произведение всех чисел
не~изменилось, и~так далее.
Всего Гриша повторил эту процедуру $k$~раз, и~все $k$~раз произведение чисел
не~менялось.
Найдите наибольшее возможное значение $k$.

\item
Про многочлен $f(x) = x^{10} + a_9 x^9 + \ldots + a_0$ известно, что
\[
    f(1) = f(-1)
\, , \quad
    \ldots
, \quad
    f(5) = f(-5)
\, .\]
Докажите, что $f(x) = f(-x)$ для любого действительного $x$.

\item
Многочлен $p(x, y)$ равен нулю во~всех целых точках.
Докажите, что он равен нулю.

\item
Докажите, что не~существует никакой (даже разрывной) функции $y = f(x)$,  что
$f(f(x)) = x^2 - 1996$ при всех $x$.

\item
Даны два различных приведённых кубических многочлена $F(x)$ и~$G(x)$.
Выписали все корни уравнений $F(x) = 0$, $G(x) = 0$ и~$F(x) = G(x)$.
Оказалось, что выписаны $8$ различных чисел.
Докажите, что наибольшее и~наименьшее из~них не~могут одновременно являться
корнями многочлена $F(x)$.

\end{problems}

