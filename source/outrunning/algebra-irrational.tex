% $date: 2015-11-25 # среда

\section*{Рациональное и иррациональное}

% $authors:
% - Глеб Погудин

\begin{problems}

\item
Докажите, что число
\quad
\subproblem $\sqrt{2} + \sqrt{3}$
\quad
\subproblem $\sqrt{2} + \sqrt{3} + \sqrt{6}$
\quad
иррационально.

\item
Докажите, что если число $(a + b \sqrt{2})$, где $a, b \in \mathbb{Q}$,
является корнем многочлена с~целыми коэффициентами, то~число $(a - b \sqrt{2})$ является корнем того~же многочлена.

\item
Число $1 + \sqrt{2} + \sqrt{3} + \sqrt{6}$ является корнем многочлена с~целыми
коэффициентами.
Найдите еще три числа, являющиеся корнями того~же многочлена.

\item
Докажите, что число $14 + 10 \sqrt{2}$ не~может быть представлено в~виде
$(a + b \sqrt{2})^2$, где $a, b \in \mathbb{Q}$.

\item
Существует~ли $\alpha$ такое, что $\cos(\alpha)$ иррационально,
а~$\cos(2 \alpha)$, $\cos(3 \alpha)$, $\cos(4 \alpha)$ и~$\cos(5 \alpha)$
рациональны?

\item
Число $\alpha = 0{,}12457{\ldots}$ определено следующим образом:
$n$-я цифра после запятой равна первой цифре слева от~запятой
в~числе~$n\sqrt{2}$.
Докажите, что $\alpha$~--- иррациональное число.

\item
Во~всех рациональных точках действительной прямой расставлены целые
числа.
Докажите, что найдется такой отрезок, что сумма чисел на~его концах
не~превосходит удвоенного числа в~его середине.

\item
Существует~ли такая сфера, на~которой имеется ровно одна рациональная
точка?

\item
Пусть $\alpha$ и~$\beta$~--- иррациональные числа, причем
$1 / \alpha + 1 / \beta = 1$.
Докажите, что последовательности $[n \alpha]$ и~$[n \beta]$ покрывают весь
натуральный ряд без перекрытий.

\end{problems}

