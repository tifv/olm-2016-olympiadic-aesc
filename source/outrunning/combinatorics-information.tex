% $date: 2015-10-28 # среда

\section*{Задачи от Андрея с подробностями}

% $authors:
% - Андрей Меньщиков
% - Глеб Погудин

\begin{problems}

\item
Даны $5$~гирь попарно различной массы.
Про любые $3$~гири $A$, $B$ и~$C$ можно спросить:
<<Верно~ли, что $m(A) < m(B) < m(C)$?>> и~узнать верный ответ.
Всегда~ли за~9 вопросов можно упорядочить все гири?

\item
Двое играют в~крестики-нолики на~клетчатой плоскости, причем за~один ход первый
ставит крестик, а~второй четыре нолика.
Первый выиграет, если ему удастся поставить три крестика в~ряд.
Может~ли второй ему помешать?

\item
Из~колоды вынули 7~карт, показали всем, перетасовали и~раздали Андрею и~Лёше
по~3 карты, а~оставшуюся карту
\\
\subproblem спрятали;
\quad
\subproblem отдали Глебу.
\\
Андрей и~Лёша могут по~очереди сообщать вслух любую информацию о~своих картах.
Могут~ли они сообщить друг другу свои карты так, чтобы при этом Глеб не~смог
вычислить местонахождение ни~одной из~тех карт, которых он не~видит?
(Андрей и~Лёша не~договаривались о~каком-либо особом способе общения;
все переговоры происходят открытым текстом).

\item
Есть $n$~гномов и~злобная Белоснежка.
Каждый день перед выходом на~работу она надевает на~каждого гнома колпак одного
из~двух цветов.
Каждый гном видит колпаки других, но~не~видит свой.
Потом все гномы одновременно говорят, какого цвета на~них колпак по~их мнению.
Если хотя~бы один ошибся~--- все идут работать на~рудники.
Как гномам договориться так, чтобы с~вероятностью $0{,}5$ не~работать?

\item
Фокусник Арутюн и~его помощник Амаяк собираются показать следующий фокус.
На~доске нарисована окружность.
Зрители отмечают на~ней $2015$ различных точек, затем помощник фокусника
стирает одну из~них.
После этого фокусник впервые входит в~комнату, смотрит на~рисунок и~отмечает
полуокружность, на~которой лежала стертая точка.
Как фокуснику договориться с~помощником, чтобы фокус гарантированно удался?

\item
Фокусник с~помощником собираются показать такой фокус.
Зритель пишет на~доске последовательность из~$N$~цифр.
Помощник фокусника закрывает две соседних цифры черным кружком.
Затем входит фокусник.
Его задача~--- отгадать обе закрытые цифры (и~порядок, в~котором они
расположены).
При каком наименьшем $N$~фокусник может договориться с~помощником так, чтобы
фокус гарантированно удался?

\item
Царь вызвал двух мудрецов.
Он дал первому 100 пустых карточек и~приказал написать на~каждой
по~положительному числу (числа не~обязательно разные), не~показывая их второму.
Затем первый может сообщить второму несколько различных чисел, каждое
из~которых либо записано на~какой-то карточке, либо равно сумме чисел
на~каких-то карточках (не~уточняя, как именно каждое число получено).
Второй должен определить, какие 100 чисел написаны на~карточках.
Если он этого не~сможет, обоим отрубят головы;
иначе из~бороды каждого вырвут столько волосков, сколько чисел сообщил первый
второму.
Как мудрецам, не~сговариваясь, остаться в~живых и~потерять минимальное
количество волосков?

\item
Придумайте, как закодировать $4$~бита информации в~сообщение из~$7$~бит, так
чтобы ошибку в~каком-то одном бите можно было выявить и~исправить.

\end{problems}

