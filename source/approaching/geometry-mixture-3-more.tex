% $date: 2016-04-06 # среда

\section*{Еще геометрия}

\begin{problems}

\item
Каждый из~двух подобных треугольников разрезали на~два треугольника так, что
одна из~получившихся частей одного треугольника подобна одной из~частей другого
треугольника.
Верно~ли, что оставшиеся части тоже подобны?

\item
Трапеция $ABCD$ и~параллелограмм $MBDK$ расположены так, что стороны
параллелограмма параллельны диагоналям трапеции.
Докажите, что площадь темно-серой части равна сумме площадей светло-серых
частей.
\begin{center}
    \jeolmfigure[width=0.4\linewidth]{areas}
\end{center}

\item
Можно~ли правильную треугольную призму разрезать на~две равные пирамиды?

\item
В~треугольнике $ABC$ угол~$A$ равен $120^{\circ}$.
Докажите, что расстояние от~центра описанной окружности до~ортоцентра равно
$AB + AC$.

\item
Внутри угла $AOD$ проведены лучи $OB$ и~$OC$, причем
$\angle AOB = \angle COD$.
В~углы $AOB$ и~$COD$ вписаны непересекающиеся окружности.
Докажите, что точка пересечения общих внутренних касательных к~этим окружностям
лежит на~биссектрисе угла $AOD$.

\item
В~остроугольном треугольнике $ABC$ проведены высоты $A A_1$ и~$B B_1$.
Докажите, что перпендикуляр, опущенный из~точки касания вписанной окружности
со~стороной~$BC$ на~прямую~$AC$, проходит через центр вписанной окружности
треугольника $A_1 B_1 C$.

\item
Существует~ли многогранник, у~которого отношение площадей любых двух граней
не~меньше 2?

\end{problems}

