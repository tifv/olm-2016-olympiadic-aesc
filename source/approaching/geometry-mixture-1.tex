% $date: 2015-10-02 # пятница

\section*{Геометрия}

% $authors:
% - Андрей Меньщиков

\begin{problems}

\item
Дан выпуклый четырехугольник $ABCD$.
Известны углы:
$\angle BCA = 40^{\circ}$, $\angle BAC = 50^{\circ}$,
$\angle BDA = 20^{\circ}$, $\angle BDC = 25^{\circ}$.
Найдите угол между диагоналями данного четырехугольника.

\item
$P$~--- точка пересечения касательных в~точках $A$ и~$B$ к~окружности~$\omega$
с~центром~$O$.
Через произвольную точку~$M$ на~отрезке~$AB$ провели прямую,
перпендикулярную $OM$.
Эта прямая пересекла прямые $PA$ и~$PB$ в~точках $C$ и~$D$.
Докажите, что $M$~--- середина отрезка~$CD$.

\item
Даны две скрещивающиеся прямые.
Все прямые, которые пересекают обе данные, красят в~красный цвет.
Найдите все точки пространства, которые останутся неокрашенными.

\item
Существует~ли выпуклая фигура, которой нельзя накрыть полукруг радиуса~1,
но~двумя копиями которой можно накрыть круг того~же радиуса?

\item
В~треугольнике $ABC$ точки $A_1$ и~$B_1$~--- середины высот, опущенных
из~вершин $A$ и~$B$, $M$ и~$S$~--- середина~$AB$ и~основание высоты
из~вершины~$C$ соответственно.
Докажите, что точки $A_1$, $B_1$, $M$, $S$ лежат на~одной окружности.

\end{problems}

