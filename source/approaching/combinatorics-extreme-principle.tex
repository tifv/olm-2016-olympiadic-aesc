% $date: 2015-09-30

\section*{Принцип крайнего}

% $authors:
% - Юлий Тихонов
%% - на основе материалов Александра Шаповалова

\begin{problems}

\itemx{$^\circ$}
\emph{Максимум или минимум.}
На~листке написаны несколько натуральных чисел.
Известно, что для любых двух найдется на~листке число, которое на~каждое из~них
делится.
Докажите, что на~листке найдется число, которое делится на~все числа.

\itemx{$^\circ$}
\emph{Первое или последнее.}
Петя разложил 10~фруктов на~две чаши весов.
Далее он~7~раз сделал такую операцию: поменял два фрукта с~правой чаши с~одним
фруктом с~левой.
Могли~ли весы быть в~равновесии вначале и~после каждой операции?

\itemx{$^\circ$}
\emph{Общий делитель.}
Пусть $x^3 + x = 5$.
Докажите, что $x$~--- иррационально.

\item
В~порядке возрастания длин лежат несколько палочек.
Можно взять любые три и~проверить, складывается~ли из~них треугольник.
За~какое наименьшее число проверок можно доказать или опровергнуть утверждение
о~том, что из~любой тройки палочек складывается треугольник?

\item
Глеб, Илья и~Витя сидят по~кругу за~столом и~едят орехи.
Сначала все орехи у~Глеба.
Он делит их~поровну между Ильей и~Витей, а~остаток (если он~есть) съедает.
Затем все повторяется: каждый следующий (по~часовой стрелке) делит имеющиеся
у~него орехи поровну между соседями, а~остаток съедает.
Вначале было больше 100 орехов.
Докажите, что хотя~бы один орех \emph{не}~будет съеден.

\item
Пусть $2^{x} = 10$.
Докажите, что $x$~--- иррационально.

\item
На~доске выписано 100 целых чисел.
Известно, что для любых пяти из~этих чисел найдутся такие шесть из~этих чисел,
что равны средние арифметические этой пятерки и~этой шестерки.
Докажите, что все числа равны.

\item
За~день в~библиотеке побывало 100 читателей, каждый по~разу.
Оказалось, что из~любых трех по~крайней мере двое там встретились.
Докажите, что библиотекарь мог сделать важное объявление в~такие два момента
времени, чтобы все 100 читателей его услышали.

\item
На~каждой клетке шахматной доски вначале стоит по~ладье.
Каждым ходом можно снять с~доски ладью, которая бьет нечетное число ладей.
Какое наибольшее число ладей можно снять?
(Ладьи бьют друг друга, если они стоят на~одной вертикали или горизонтали
и~между ними нет других ладей).

\end{problems}

