% $date: 2016-01-13 # среда

\section*{Соответствия}

% $authors:
% - Виктор Трещёв

\begin{problems}

\item
На~параллельных прямых $a$ и~$b$ отмечены точки $A_1, \ldots, A_n$
и~$B_1, \ldots, B_k$.
Проведены все отрезки $A_i B_j$.
Оказалось, что никакие три из~них не~пересекаются в~одной точке.
Сколько всего точек пересечения у~отрезков~$A_i B_j$?

\item
Номер автобусного билета состоит из~$6$ цифр.
Билет называют счастливым, если сумма первых трех цифр его номера равна сумме
трех последних цифр.
Докажите,что сумма номеров счастливых билетов делится на~$13$.

\item
Каких автобусных билетов больше: счастливых или тех, чьи номера делятся на~$11$?

\item
На~окружности даны $2015$ точек, одна из~них отмечена.
Рассмотрим всевозможные выпуклые многоугольники с~вершинами в~этих точках.
Каких многоугольников больше: тех, которые содержат отмеченную точку, или тех,
которые ее не~содержат?

\item
Двое бросают монетку: один бросил ее $10$~раз, другой~--- $11$.
Чему равна вероятность того, что у~второго монета упала орлом большее число
раз, чем у~первого?

\item
Имеется сто палочек длины $1, 2, \ldots, 100$.
Наудачу выбирается три из~них.
Что больше~--- вероятность того, что из~них можно составить треугольник или
вероятность того, что нельзя?

\item
На~собрание пришло $n$~человек ($n > 1$).
Оказалось, что у~любых двух из~них есть среди собравшихся ровно два других
общих знакомых.
\\
\subproblem
Докажите, что каждый из~них знаком с~одинаковым числом людей на~этом собрании.
\\
\subproblem
Покажите, что $n$ может быть больше $4$.

\end{problems}

