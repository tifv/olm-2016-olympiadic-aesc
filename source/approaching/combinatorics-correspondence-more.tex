% $date: 2016-01-20 # среда

\section*{Соответствия. Добавка}

% $authors:
% - Виктор Трещёв

\begin{problems}

\item
\subproblem
Какое наибольшее число полей на~доске $8 \times 8$ можно закрасить в~черный
цвет так, чтобы в~каждом уголке из~трех полей было по~крайней мере одно
незакрашенное поле?
\\
\subproblem
Какое наименьшее число полей на~доске $8 \times 8$ можно закрасить в~черный
цвет так, чтобы в~каждом уголке из~трех полей было по~крайней мере одно черное
поле?

\item
Существует~ли натуральное число, у~которого нечетное количество четных
натуральных делителей и~четное количество нечетных?

\item
У~Пети есть $12$ одинаковых разноцветных вагончиков (некоторые, возможно,
одного цвета, но~неизвестно, сколько вагончиков какого цвета).
Петя считает, что различных $12$-вагонных поездов он сможет составить больше,
чем $11$-вагонных.
Не~ошибается~ли Петя?
(Поезда считаются одинаковыми, если в~них на~одних и~тех~же местах находятся
вагончики одного и~того~же цвета.)

\item
Петя подсчитал количество всех возможных $m$-буквенных слов, в~записи которых
могут использоваться только четыре буквы $T$, $O$, $W$ и~$N$, причем в~каждом
слове букв $T$ и~$O$ поровну.
Вася подсчитал количество всех возможных $2m$-буквенных слов, в~записи которых
могут использоваться только две буквы $T$ и~$O$, и~в~каждом слове этих букв
поровну.
У~кого слов получилось больше?
(Слово~--- это любая последовательность букв.)

\end{problems}

