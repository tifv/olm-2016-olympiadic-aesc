% $date: 2016-03-16 # среда

\section*{Подготовка к олимпиаде по геометрии}

\begin{problems}

\item
Докажите, что любой жесткий плоский треугольник~$T$ площади меньше четырех
можно просунуть сквозь треугольную дырку~$Q$ площади~$3$.

\item
Прямая~$a$ пересекает плоскость~$\alpha$.
Известно, что в~этой плоскости найдутся $2011$ прямых, равноудаленных от~$a$
и~не~пересекающих $a$.
Верно~ли, что $a$ перпендикулярна $\alpha$?

\item
$ABCDE$~-- правильный пятиугольник.
Точка~$B'$ симметрична точке~$B$ относительно прямой~$AC$.
Можно~ли пятиугольниками, равными $AB'CDE$, замостить плоскость?

\item
Существует~ли выпуклый пятиугольник, в~котором каждая диагональ равна какой-то
стороне?

\item
На~плоскости проведены шесть прямых.
Известно, что для любых трех из~них найдется четвертая из~этого~же набора
прямая, такая что все четыре будут касаться одной окружности.
Обязательно~ли все шесть прямых касаются одной и~той~же окружности?

\item
Шесть отрезков таковы, что из~любых трех можно составить треугольник.
Верно~ли, что из~этих отрезков можно составить тетраэдр?

\item
Четырехугольник $ABCD$ вписан в~окружность, центр~$O$ которой лежит внутри
него.
Касательные к~окружности в~точках $A$ и~$C$ и~прямая, симметричная $BD$
относительно точки~$O$, пересекаются в~одной точке.
Докажите, что произведения расстояний от~$O$ до~противоположных сторон
четырехугольника равны.

\item
Середины противоположных сторон шестиугольника соединены отрезками.
Оказалось, что точки попарного пересечения этих отрезков образуют
равносторонний треугольник.
Докажите, что проведенные отрезки равны.

\item
Пусть $A A_1$, $B B_1$, $C C_1$~--- высоты неравнобедренного остроугольного
треугольника $ABC$;
окружности описанные около треугольников $ABC$ и~$A_1 B_1 C$, вторично
пересекаются в~точке~$P$, $Z$~--- точка пересечения касательных к~описанной
окружности треугольника $ABC$, проведенных в~точках $A$ и~$B$.
Докажите, что прямые $AP$, $BC$ и~$Z C_1$ пересекаются в~одной точке.

\end{problems}

