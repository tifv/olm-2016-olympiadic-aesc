% $date: 2016-01-22 # пятница

\section*{Разнобой по геометрии}

% $authors:
% - Леонид Попов
% - Артемий Соколов

\begin{problems}

\item
Пусть точки $A$, $B$, $C$ лежат на~окружности, а~прямая~$b$ касается этой
окружности в~точке~$B$.
Из~точки~$P$, лежащей на~прямой~$b$, опущены перпендикуляры $P A_1$ и~$P C_1$
на~прямые $AB$ и~$BC$ соответственно (точки $A_1$ и~$C_1$ лежат
на~отрезках $AB$ и~$BC$).
Докажите, что $A_1 C_1 \perp AC$.

\item
Окружность, вписанная в~прямоугольный треугольник $ABC$ с~гипотенузой~$AB$,
касается сторон $AB$, $BC$ и~$AC$ в~точках $C_1$, $A_1$ и~$B_1$ соответственно.
Пусть $B_1 H$~--- высота треугольника $A_1 B_1 C_1$.
Докажите, что точка~$H$ лежит на~биссектрисе угла $CAB$.

\item
Дан прямоугольный треугольник $ABC$ с~прямым углом~$C$.
Пусть $BK$~--- биссектриса этого треугольника.
Окружность, описанная около треугольника $AKB$, пересекает вторично
сторону~$BC$ в~точке~$L$.
Докажите, что $CB + CL = AB$.

\item
В~треугольнике $ABC$ проведены биссектрисы $AD$, $BE$ и~$CF$, пересекающиеся
в~точке~$I$.
Серединный перпендикуляр к~отрезку~$AD$ пересекает прямые $BE$ и~$CF$
в~точках $M$ и~$N$ соответственно.
Докажите, что точки $A$, $I$, $M$ и~$N$ лежат на~одной окружности.

%\item
%Окружности $\omega_1$ и~$\omega_2$ пересекаются в~точках $D$ и~$P$.
%Точки $A$ и~$B$ лежат на~окружностях $\omega_1$ и~$\omega_2$ соответственно,
%причем $AB$~--- общая касательная к~этим окружностям, а~$D$ лежит внутри
%треугольника $ABP$.
%$AD$ вторично пересекает окружность~$\omega_2$ в~точке~$C$,
%$M$~--- середина $BC$.
%Докажите, что $\angle DPM = \angle BDC$.

\item
На~стороне~$BC$ выпуклого четырехугольника $ABCD$ взяты точки $E$ и~$F$
(точка~$E$ ближе к~точке~$B$, чем точка~$F$).
Известно, что $\angle BAE = \angle CDF$ и~$\angle EAF = \angle FDE$.
Докажите, что $\angle FAC = \angle EDB$.

\item
В~неравнобедренном треугольнике $ABC$ провели биссектрисы угла $ABC$ и~угла,
смежного с~ним.
Они пересекли прямую~$AC$ в~точках $B_1$ и~$B_2$ соответственно.
Из~точек $B_1$ и~$B_2$ провели касательные к~окружности,
вписанной в~треугольник $ABC$, отличные от~прямой~$AC$.
Они касаются этой окружности в~точках $K_1$ и~$K_2$ соответственно.
Докажите, что точки $B, K_1$ и~$K_2$ лежат на~одной прямой.

\end{problems}

