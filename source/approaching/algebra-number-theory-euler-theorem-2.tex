% $date: 2015-11-25 # среда

\section*{Теорема Эйлера}

% $authors:
% - Виктор Трещёв

\begingroup
    \ifx\mathup\undefined
        \def\eulerphi{\upphi}
    \else
        \def\eulerphi{\mathup{\phi}}
    \fi
    \providecommand\divides{\mathrel{\vert}}

\begin{problems}

\item
Решите уравнения:
\\
\subproblem $\eulerphi(x) = 2$;
\qquad
\subproblem $\eulerphi(x) = 8$;
\qquad
\subproblem $\eulerphi(x) = 12$;
\qquad
\subproblem $\eulerphi(x) = 14$.

\item
Решите уравнения:
\\
\subproblem $\eulerphi(x) = x / 2$;
\qquad
\subproblem $\eulerphi(x) = x / 3$;
\qquad
\subproblem $\eulerphi(x) = x / 4$.

\item
Известно, что $(m, n) > 1$.
Что больше: $\eulerphi(m \cdot n)$ или $\eulerphi(m) \cdot \eulerphi(n)$?

\item
Докажите \emph{тождество Гаусса:}
\[
    \sum_{d \divides n}
        \eulerphi(d)
=
    n
\, . \]

\item
Существует~ли степень тройки, заканчивающаяся на~$0001$?

\item
Докажите, что для любого нечетного числа~$m$ существует такое натуральное
$n$, что $(2^n - 1) \kratno m$.

\item
Найдите все такие целые числа~$a$, для которых число $a^{10} + 1$ делится
на~$10$.

\end{problems}

\endgroup % \def\eulerphi \def\divides

