% $date: 2016-02-24 # среда

\section*{ММО. Комбинаторная геометрия}

% $authors:
% - Виктор Трещёв

\begin{problems}

\item
Будем называть \emph{змейкой} ломаную, у~которой все углы между соседними
звеньями равны, причем для любого некрайнего звена соседние с~ним звенья лежат
в~разных полуплоскостях от~этого звена.
Барон Мюнхгаузен заявил, что отметил на~плоскости 6~точек и~нашел 6 разных
способов соединить их (пятизвенной) змейкой (вершины каждой из~змеек~---
отмеченные точки).
Могут~ли его слова быть правдой?

\item
Дан треугольник, у~которого нет равных углов.
Петя и~Вася играют в~такую игру: за~один ход Петя отмечает точку на~плоскости,
а~Вася красит ее по~своему выбору в~красный или синий цвет.
Петя выиграет, если какие-то три из~отмеченных им и~покрашенных Васей точек
образуют одноцветный треугольник, подобный исходному.
За~какое наименьшее число ходов Петя сможет гарантированно выиграть
(каков~бы ни~был исходный треугольник)?

\item
Назовем точку на~плоскости \emph{узлом,} если обе ее координаты~--- целые
числа.
Дан треугольник с~вершинами в~узлах, внутри него расположено не~меньше двух
узлов.
Докажите, что среди узлов внутри треугольника можно выбрать такие два узла, что
проходящая через них прямая содержит одну из~вершин треугольника или
параллельна одной из~сторон треугольника.

\item
Клетки бесконечного клетчатого листа бумаги раскрасили в~черный и~белый цвета
в~шахматном порядке.
Пусть $X$~--- треугольник площади~$S$ с~вершинами в~узлах сетки.
Покажите, что есть такой подобный $X$ треугольник с~вершинами в~узлах сетки,
что площадь его белой части равна площади черной части и~равна $S$.

\end{problems}

