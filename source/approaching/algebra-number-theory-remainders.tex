% $date: 2015-11-18 # среда

\section*{Немного энтомологии}

% $authors:
% - Виктор Трещёв

На~окружности отмечено $n$~точек.
В~одной из~точек сидит кузнечик.
Он~умеет прыгать по~часовой стрелке на~$b$~точек
(через $(b - 1)$ точек на~$b$-ю).

\begin{problems}

\itemy{0}
Докажите, что через некоторое количество прыжков кузнечик окажется в~точке,
в~которой уже бывал.

\item
Докажите, что первая точка, в~которой кузнечик побывает дважды~--- это начало
пути кузнечика.

\item
В~отмеченных точках растет травка.
Сколько травки сможет съесть кузнечик?
Каким должно быть число~$b$, чтобы кузнечик съел всю травку на~окружности?

\end{problems}

\definition
Пусть есть два целых числа $a$ и~$b$ и~натуральное число $n$.
Числа $a$ и~$b$ называются \emph{сравнимыми по~модулю} $n$, если
$(a - b)$ делится на~$n$.
Пишут
\[
    a \equiv b \pmod n
\,.\]
Если $b$~--- число от~$0$ до~$n - 1$, то~говорят, что $a$
\emph{имеет остаток~$b$ по~модулю~$n$}.

\begin{problems}

\item
Докажите утверждения: \emph{(пункты сдаются одновременно)}
\\
\subproblem $a \equiv a \pmod n$; %\quad\emph{(рефлексивность)}
\\
\subproblem
\(
    a \equiv b \pmod n
\;\Rightarrow\;
    b \equiv a \pmod n
\);
%\quad\emph{(симметричность)}
\\
\subproblem
\(
    a \equiv b \pmod n \text{ и } b \equiv c \pmod n
\;\Rightarrow\;
    a \equiv c \pmod n
\);
%\quad\emph{(транзитивность)}
\\
\subproblem
\(
    a \equiv b \pmod n \text{ и } c \equiv d \pmod n
\;\Rightarrow\;
    a + c \equiv b + d \pmod n
\);
%(остатки можно складывать\ldots)
\\
\subproblem
\(
    a \equiv b \pmod n \text{ и } c \equiv d \pmod n
\;\Rightarrow\;
    a c \equiv b d \pmod n
\);
%(\ldotsи умножать\ldots)
\\
\subproblem
В~каких случаях
\(
    a c \equiv b c \pmod n
\;\Rightarrow\;
    a\equiv b \pmod n
\)?
% (\ldots а вот делить (сокращать) можно не всегда).

\item
Для каких $a$ и~$b$ у~сравнения $ax \equiv b \pmod n$ найдется решение $x$?
Опишите все решения этого сравнения.

\item
Найдите остаток от~деления $3^{2015}$ на~$14$.

\item
Докажите, что $1^{101} + 2^{101} + \dots + 2015^{101}$ делится на~$1008$.

\item
Докажите, что $11^{n+2} + 12^{2n + 1}$ делится на~$133$ при любом
натуральном~$n$.

\item
Докажите, что для любого числа $d$, не~делящегося на~$2$ и~на~$5$, найдется
число, в~десятичной записи которого содержатся одни единицы, и~которое
делится на~$d$.

\item\claim{Теорема Вильсона}
Докажите, что для простого~$p$ выполнено сравнение
$(p - 1)! \equiv (-1) \pmod p$.

\item
Докажите, что ни~одно из~чисел вида $10^{3n+1}$ нельзя представить в~виде суммы
двух кубов натуральных чисел.

\end{problems}

