% $date: 2016-02-03 # среда

\section*{Разнобой по региону}

% $authors:
% - Виктор Трещёв

\begin{problems}

\item
Тридцать девочек~--- $13$ в~красных платьях и~$17$ в~синих платьях~--- водили
хоровод вокруг новогодней ёлки.
Впоследствии каждую из~них спросили, была~ли ее соседка справа в~синем платье.
Оказалось, что правильно ответили те и~только те девочки, которые стояли между
девочками в~платьях одного цвета.
Сколько девочек могли ответить утвердительно?

\item
Дан прямоугольный треугольник $ABC$ с~прямым углом~$C$.
Пусть $BK$~--- биссектриса этого треугольника.
Окружность, описанная около треугольника $AKB$, пересекает вторично
сторону~$BC$ в~точке~$L$.
Докажите, что $CB + CL = AB$.

\item
Найдите все числа~$a$ такие, что для любого натурального~$n$ число
$a n (n + 2) (n + 4)$ будет целым.

\item
В~языке племени АУ две буквы~--- <<a>> и~<<y>>.
Некоторые последовательности этих букв являются словами, причем в~каждом слове
не~больше 13~букв.
Известно, что если написать подряд любые два слова, то~полученная
последовательность букв не~будет словом.
Найдите максимальное возможное количество слов в~таком языке.

\item
На~доске написано уравнение $x^3 + *x^2 + *x + * = 0$.
Петя и~Вася по~очереди заменяют звездочки на~рациональные числа: вначале Петя
заменяет любую из~звездочек, потом Вася~--- любую из~двух оставшихся, а~затем
Петя~--- оставшуюся звездочку.
Верно~ли, что при любых действиях Васи Петя сможет получить уравнение,
у~которого разность каких-то двух корней равна $2016$?

\item
Натуральное число~$m$ таково, что сумма цифр в~десятичной записи числа~$2^m$
равна $8$.
Может~ли при этом последняя цифра числа~$2^m$ быть равной $6$?

\item
Прямые, касающиеся окружности~$\omega$ в~точках $B$ и~$D$, пересекаются
в~точке~$P$.
Прямая, проходящая через $P$, высекает на~окружности хорду~$AC$.
Через произвольную точку отрезка~$AC$ проведена прямая, параллельная $BD$.
Докажите, что она делит длины ломаных $ABC$ и~$ADC$ в~одинаковых
отношениях.

\item
Все клетки квадратной таблицы $n \times n$ пронумерованы в~некотором порядке
числами от~$1$ до~$n^2$.
Петя делает ходы по~следующим правилам.
Первым ходом он ставит ладью в~любую клетку.
Каждым последующим ходом Петя может либо поставить новую ладью на~какую-то
клетку, либо переставить ладью из~клетки с~номером~$a$ ходом по~горизонтали или
по~вертикали в~клетку с~номером большим, чем $a$.
Каждый раз, когда ладья попадает в~клетку, эта клетка немедленно закрашивается;
ставить ладью на~закрашенную клетку запрещено.
Какое наименьшее количество ладей потребуется Пете, чтобы независимо
от~исходной нумерации он смог за~несколько ходов закрасить все клетки таблицы?

\item
Докажите, что найдется такое натуральное число $n > 1$, что произведение
некоторых $n$ последовательных натуральных чисел равно произведению
некоторых $n + 100$ последовательных натуральных чисел.

\end{problems}

