% $date: 2015-10-21 # среда

\section*{Оценка + пример}

% $authors:
% - Виктор Трещёв

\begin{problems}

\item
Среднее арифметическое десяти различных натуральных чисел равно 15.
Найдите наибольшее значение наибольшего из~этих чисел.

\item
Какое наименьшее количество клеток нужно отметить на~шахматной доске, чтобы
\\
\emph{(1)}\enspace
среди отмеченных клеток не~было соседних
(имеющих общую сторону или общую вершину),
\\
\emph{(2)}\enspace
добавление к~этим клеткам любой одной клетки нарушало пункт~(1)?

\item
Какое наибольшее число белых и~черных фишек можно расставить на~шахматной доске
так, чтобы на~каждой горизонтали и~на~каждой вертикали белых фишек было ровно
в~два раза больше, чем черных?

\item
Город имеет форму квадрата $5 \times 5$;
по границам между клеточками проходят улицы.
Какую наименьшую длину может иметь маршрут, если нужно пройти по~каждой улице
этого города и~вернуться в~прежнее место?
(По~каждой улице можно проходить любое число раз.)

%\begin{picture}(100, 60)
%\multiput(50, 0)(0, 10){6}{\line(1, 0){50}}
%\multiput(50, 0)(10, 0){6}{\line(0, 1){50}}
%\end{picture}

\item
Петя красит клетки таблицы $n \times n$ по~следующему правилу: если какая-то
незакрашенная клетка граничит по~стороне с~двумя закрашенными, то~ее можно
закрасить.
Какое наименьшее число клеток могло быть закрашено изначально, если известно,
что Петя смог закрасить все келтки?

\item
Петя красит клетки таблицы $n \times m$ по~следующему правилу: если
в~каком-либо квадрате $2 \times 2$ уже закрашены три клетки, то~он может
закрасить четвертую.
Какое наименьшее число клеток могло быть закрашено изначально, если известно,
что Петя смог закрасить все келтки?

\item
На~новом сайте зарегистрировалось 2000 человек.
Каждый пригласил к~себе в~друзья по~1000 человек.
Два человека объявляются друзьями тогда и~только тогда, когда каждый из~них
пригласил другого в~друзья.
Какое наименьшее количество пар друзей могло образоваться?

\end{problems}

