% $date: 2015-11-11 # среда

\section*{Оценка + пример. Добавка}

% $authors:
% - Виктор Трещёв

\begin{problems}

\item
Число называется \emph{несложным,} если оно является произведением ровно двух
простых (быть может, равных).
Какое наибольшее количество несложных чисел может идти подряд?

\item
Какое максимальное число королей, не~бьющих друг друга, можно расставить
на~шахматной доске $8 \times 8$?

\item
У~Чебурашки есть набор из~36 камней массами
$1\,\text{г}, 2\,\text{г}, \ldots, 36\,\text{г}$,
а~у~Шапокляк есть суперклей, одной каплей которого можно склеить два камня
в~один (соответственно, можно склеить 3~камня двумя каплями и~так далее).
Шапокляк хочет склеить камни так, чтобы Чебурашка не~смог из~получившегося
набора выбрать один или несколько камней общей массой $37\,\text{г}$.
Какого наименьшего количества капель клея ей хватит, чтобы осуществить
задуманное?

\item
В~Черноморске живет $n$~жителей.
Все они образовали партию.
Затем эта партия разделилась на~две непересекающиеся фракции, каждая из~которых
объявила себя партией.
Каждый последующий день каждая из~образовавшихся в~предыдущий день партий
делилась на~две фракции, а~каждая фракция, в~которой больше одного человека,
тут~же объявляла себя партией.
Иные партии не~образовывались.
Когда процесс закончился, каждый житель заплатил членский взнос в~1~рубль
каждой партии, в~которой состоял.
Найдите максимально возможную сумму взносов.

\item
Глеб кладет спички в~клеточки таблицы $5 \times 5$.
Каждая спичка должна лежать полностью внутри одной из~клеточек.
Длина каждой спички равна длине диагонали клеточки.
Спички не~могут пересекаться (в~том числе соприкасаться концами).
Какое наибольшее количество спичек может выложить Глеб?

\end{problems}

