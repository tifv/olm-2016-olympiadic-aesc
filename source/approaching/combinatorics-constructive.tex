% $date: 2015-09-23 # среда

\section*{Конструктивы}

% $authors:
% - Виктор Трещёв

\begin{problems}

\item
Нарисуйте фигуру, которую можно разрезать на~четыре фигурки, изображенные
слева, а~можно~--- на~пять фигурок, изображенных справа.
(Фигурки можно поворачивать.)
\begin{center}
    \jeolmfigure[width=0.1\linewidth]{pentomino-T}
\qquad
    \jeolmfigure[width=0.1\linewidth]{tetramino-T}
\end{center}

\item
На~шахматной доске $8 \times 8$ стоит кубик
(нижняя грань совпадает с~одной из~клеток доски).
Его прокатили по~доске, перекатывая через ребра, так, что кубик побывал на~всех
клетках (на~некоторых, возможно, несколько раз).
Могло~ли случиться, что одна из~его граней ни~разу не~лежала на~доске?

\item
На~плоскости нарисован черный квадрат.
Имеется семь квадратных плиток того~же размера.
Можно~ли расположить их на~плоскости так, чтобы они не~перекрывались и~чтобы
каждая плитка покрывала хотя~бы часть черного квадрата?

\item
Существуют~ли натуральные числа $m$ и~$n$, для которых верно равенство:
$(-2 a^n b^n)^m + (3 a^m b^m)^n = a^6 b^6$?

\item
Tреугольник разбили на~пять треугольников, ему подобных.
Bерно~ли, что исходный треугольник -- прямоугольный?

\item
Шесть отрезков таковы, что из~любых трех можно составить треугольник.
Bерно~ли, что из~этих отрезков можно составить тетраэдр?

\item
Существуют~ли такие натуральные числа $a$, $b$, $c$, $d$, что
$a^3 + b^3 + c^3 + d^3 = 100^{100}$?

\item
\subproblem
В~бесконечной последовательности бумажных прямоугольников площадь $n$-го
прямоугольника равна $n^2$.
Обязательно~ли можно покрыть ими плоскость?
Наложения допускаются.
\\
\subproblem
Дана бесконечная последовательность бумажных квадратов.
Обязательно~ли можно покрыть ими плоскость (наложения допускаются), если
известно, что для любого числа~$N$ найдется набор квадратов суммарной площади
больше~$N$?

\end{problems}

