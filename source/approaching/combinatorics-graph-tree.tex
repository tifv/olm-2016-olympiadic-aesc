% $date: 2016-04-20 # среда

\section*{Деревья}

\emph{Дерево}~--- связный граф без циклов.

\begin{problems}

\item
Докажите, что при удалении любого ребра из~дерева оно превращается в~несвязный
граф.

\item
Докажите, что из~связного графа можно выкинуть несколько ребер так, чтобы
осталось дерево.

\item
Докажите, что в~дереве с~$n$~вершинами ровно $(n - 1)$ ребер.

\item
Вершина называется \emph{висячей,} если из~нее выходит ровно одно ребро.
Докажите, что в~дереве не~меньше двух висячих вершин.

\item
Докажите, что в~связном графе из~$n$ вершин не~меньше $(n - 1)$ ребер.

%\item
%Докажите, что если в~связном графе $n$~вершин и~$(n - 1)$ ребер, то~он~---
%дерево.

%\item
%Волейбольная сетка имеет вид прямоугольника размером $50 \times 600$ клеток.
%Какое наибольшее число веревочек можно перерезать так, чтобы сетка не~распалась
%на~куски?

%\item
%Докажите, что в~любом связном графе можно удалить вершину вместе со~всеми
%выходящими из~нее ребрами так, чтобы он остался связным.

\item
Докажите, что для любого набора чисел
\(
    0 < d_1 \leq d_2 \leq \ldots \leq d_n
\)
такого, что $d_1 + \ldots + d_n = 2 n - 2$, найдется дерево, где степени вершин
будут $d_1, \ldots, d_n$.

\item
А~единственно~ли дерево из~предыдущей задачи?

\item
Может~ли у~графа быть ровно два остовных дерева?

\item
В~группе каждый имеет знакомого.
Докажите, что эту группу можно разбить на~две так, чтобы каждый человек имел
знакомого из~другой группы.

\item
В~дереве все вершины были занумерованы числами от~$1$ до~$n$.
Нумерацию поменяли, но~оказалось, что если вершины $i$ и~$j$ смежны, то~они
и~раньше были смежны.
Докажите, что найдется либо вершина, номер которой не~изменился, либо ребро,
у~которого набор номеров концов остался таким~же.

\item
В~графе есть остовное дерево с~$m$ висячими вершинами и~остовное дерево с~$n$
висячими вершинами.
Докажите, что для всякого $k$ такого, что $m < k < n$, найдется остовное дерево
с~$k$ висячими вершинами.

%\item
%В~стране $n$~городов, между некоторыми есть дороги.
%Известно, что из~каждого города можно попасть в~каждый, причем из~каждого
%города выходит не~более $d$~дорог.
%Докажите, что всю страну можно разделить на~два региона так, что в~каждом
%регионе можно будет из~любого города попасть в~любой и~размер каждого региона
%будет не~меньше $\lfloor \frac{n - 1}{d} \rfloor$.

%\item
%Имеется многоугольник с~$n$ вершинами.
%Докажите, что в~нем найдется диагональ, лежащая целиком внутри него, такая, что
%после ее проведения получаются многоугольники с~не~менее чем $(n / 3 - 1)$
%сторонами.
%(Можно без доказательства пользоваться тем фактом, что в~любом многоугольнике
%найдется диагональ, лежащая внутри.)

%\item
%В~связном графе на~$2n$ вершинах все вершины имеют четную степень.
%Докажите, что количество остовных деревьев в~нем четно.

\end{problems}

