% $date: 2015-10-09 # пятница

\section*{Ищем окружности}

% $authors:
% - Леонид Попов

\begin{problems}

\item
$AH$~--- высота остроугольного треугольника $ABC$, $K$ и~$L$~--- основания
перпендикуляров, опущенных из~точки $H$ на~стороны $AB$ и~$AC$.
Докажите, что точки $B$, $K$, $L$ и~$C$ лежат на~одной окружности.

\item
На~плоскости даны прямая~$\ell$ две точки $A$ и~$B$ по~одну сторону от~прямой.
На~прямой~$\ell$ выбрана точка~$M$, сумма расстояний от~которой до~$A$ и~$B$
наименьшая, и~точка~$N$, для которой расстояния до~точек $A$ и~$B$ равны:
$AN = BN$.
Докажите, что точки $A$, $B$, $M$, $N$ лежат на~одной окружности.

\item
В~окружности проведены две пересекающиеся хорды $AB$ и~$CD$.
На~отрезке~$AB$ взяли точку~$M$ так, что $AM = AC$, а~на~отрезке~$CD$~---
точку~$N$ так, что $DN = DB$.
Докажите, что если точки $M$ и~$N$ не~совпадают, то~прямая~$MN$ параллельна
прямой~$AD$.

\item
\subproblem
Докажите, что точка, симметричная ортоцентру~$H$ треугольника $ABC$
относительно середины стороны, лежит на~описанной окружности
треугольника $ABC$.
\\
\subproblem
Докажите, что $A$, $C$, $H$ и~проекция $H$ на~медиану треугольника, выходящую
из~вершины~$B$, лежат на~одной окружности.

\item
Дан треугольник $ABC$.
На~продолжениях сторон $AB$ и~$CB$ за~точку~$B$ взяты точки $C_1$ и~$A_1$
соответственно так, что $AC = A_1 C = A C_1$.
Докажите, что описанные окружности треугольников $A B A_1$ и~$C B C_1$
пересекаются на~биссектрисе угла~$B$.

\item
В~треугольнике $ABC$ на~стороне~$BC$ выбрана точка~$M$ так, что точка
пересечения медиан треугольника $ABM$ лежит на~описанной окружности
треугольника $ACM$, а~точка пересечения медиан треугольника $ACM$ лежит
на~описанной окружности треугольника $ABM$.
Докажите, что медианы треугольников $ABM$ и~$ACM$ из~вершины~$M$ равны.

\end{problems}

