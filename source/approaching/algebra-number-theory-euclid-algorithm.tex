% $date: 2015-11-18 # среда

\section*{Самый главный алгоритм}

% $authors:
% - Виктор Трещёв

\begin{problems}

\item
Может~ли наибольший общий делитель двух натуральных чисел быть больше
их~разности?

\item
Каков наибольший возможный общий делитель чисел $9 m + 7 n$ и $3 m + 2 n$, если
числа $m$ и~$n$ не~имеют общих делителей, кроме единицы?

\item
Можно~ли с~помощью циркуля и~линейки разделить угол $19^\circ$ на~$19$ равных
частей?

\item
Найдите $\text{НОД}(11\ldots1, 11\ldots1)$.
Здесь в~первом числе $100$ единиц, а~во~втором~--- $60$.

\item
Найдите $\text{НОД}(x^{n} - 1, x^{m} - 1)$.

\item
Даны два многочлена $P(x)$ и~$Q(x)$.
Для каких многочленов $R(x)$ разрешимо следующее уравнение?
\[
    P(x) U(x) + Q(x) V(x) = R(x)
\]
($U(x)$ и $V(x)$~--- переменные многочлены.)

\item
Найдите наименьшее натуральное число, не~представимое в~виде
\[
    \frac{2^a - 2^b}{2^c - 2^d}
\; , \]
где $a$, $b$, $c$, $d$~--- натуральные числа.

\item
Есть шоколадка в~форме равностороннего треугольника со~стороной~$n$,
разделенная бороздками на~равносторонние треугольники со~стороной~$1$.
Играют двое.
За~ход можно отломать от~шоколадки треугольный кусок вдоль бороздки, съесть
его, а~остаток передать противнику.
Тот, кто получит последний кусок~--- треугольник со~стороной $1$,~---
победитель.
Тот, кто не~может сделать ход, досрочно проигрывает.
Кто выигрывает при правильной игре?

\end{problems}

