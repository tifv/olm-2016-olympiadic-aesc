% $date: 2016-04-27 # среда

\section*{Квадратим квадратные квадраты}

\begingroup
    \def\abs#1{\lvert #1 \rvert}%
    \def\divides{\mathrel{\vert}}%
    \def\imag{\mathrm{i}}%

\theorem
Натуральное число~$n$ представимо в~виде суммы двух квадратов тогда и~только
тогда, когда все его простые делители вида $4k + 3$ входят в~четных степенях.

\claim{Лемма 1}
Пусть $p > 2$~--- простое число.
Сравнение $x^2 \equiv -1 \pmod{p}$ разрешимо тогда и~только тогда, когда
$p = 4k + 1$.

\begin{problems}

\item
Пусть $x$~--- остаток по~модулю~$p$.
Рассмотрим четверку чисел $x, -x, x^{-1}, -x^{-1}$.
Докажите, что различные четверки не~пересекаются.

\item
Бывает~ли так, что внутри четверки некоторые числа совпадают?
В~каких случаях это может произойти?
Рассмотрите все варианты.

\item
Посчитайте все четверки чисел по~модулю $p$ для случаев $p = 4 k + 1$
и~$p = 4 k + 3$.
Докажите лемму~1.

\end{problems}

\claim{Лемма 2}
Пусть $p = 4 k + 1$.
Тогда при некоторых $a$ и~$b$ выполняется $p = a^2 + b^2$.

Пусть $s^2 \equiv -1 \pmod{p}$, $M = \{ 0, 1, 2, \ldots, [\sqrt{p}] \}$,
$x, y \in M$.

\begin{problems}

\item
Докажите, что количество различных пар чисел $(x, y)$ больше $p$.

\item
Докажите, что при некоторых $x_1$, $y_1$, $x_2$, $y_2$ выполнено
$x_1 + s y_1 \equiv x_2 + s y_2 \pmod{p}$.

\item
Пусть $a = x_1 - x_2$, $b = y_1 - y_2$.
Докажите, что $a^2 + b^2 \equiv 0 \pmod p$.

\item
Докажите, что $a^2 + b^2 = p$.

\end{problems}

\claim{Лемма 3}
Пусть некоторые $m$, $n$ представимы  в~виде суммы двух квадратов.
Тогда их произведение $m \cdot n$ тоже представимо.

\begin{problems}

\item
Рассмотрим два комплексных числа $z_1 = a_1 + \imag b_1$
и~$z_2 = a_2 + \imag b_2$.
Чему равно их произведение?
Чему равно произведение $\abs{z_1}^2 \cdot \abs{z_2}^2$?
Докажите лемму~3.

\end{problems}

\claim{Лемма 4}
Пусть $n = a^2 + b^2$, $p = 4 k + 3$, $p \divides n$.
Тогда $p \divides a$ и~$p \divides b$.

\begin{problems}

\item
Воспользуйтесь леммой~1 и~докажите лемму~4.

\item \claim{Следствие}
Пусть $n = a^2 + b^2$, $p = 4 k +3$, $p \divides n$.
Тогда $p^2 \divides n$.

\item
При помощи лемм 2--4 докажите теорему.

\end{problems}

\endgroup % \def\abs \def\divides \def\imag

