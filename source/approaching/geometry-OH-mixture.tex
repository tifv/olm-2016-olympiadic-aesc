% $date: 2016-01-15 # пятница

\section*{Гидроксид разнобоя}

% $authors:
% - Юлий Тихонов

\subsection*{Упражнения}

В~треугольнике $ABC$ отмечены центр описанной окружности $O$
и~точка пересечения высот \emph{(ортоцентр)}~$H$.

\begingroup
    \let\vect\overline

\begin{problems}

\item \claim{Лемма 1}
%В~треугольнике $ABC$ обозначены ортоцентр~$H$ и~центр описанной окружности $O$.
Докажите, что $\angle BAO = \angle CAH$.

\item \claim{Лемма 2}
Докажите равенство векторов $\vect{OH} = \vect{OA} + \vect{OB} + \vect{OC}$.
\\
Эквивалентно, $\vect{AH} = \vect{OB} + \vect{OC}$.

\item
Восстановите треугольник по~данным $A$, $O$ и~$H$.

\item
В~остроугольном треугольнике $ABC$ проведена биссектриса~$AD$.
Перпендикуляр, опущенный из~$B$ на~прямую~$AD$, пересекает описанную окружность
треугольника $ABD$ в~точке~$E$, отличной от~$B$.
Докажите, что точки $A$, $E$ и~центр описанной окружности~$O$
треугольника $ABC$ лежат на~одной прямой.

\item
\subproblem
Пусть в~треугольнике $ABC$ точка~$O$~--- центр описанной окружности.
Пусть $B'$ и~$C'$~--- образы точек $B$ и~$C$ соответственно при некоторой
инверсии с~центром в~$A$.
Докажите, что $AO$~--- прямая, содержащая высоту в~треугольнике $AB'C'$.
\\
\subproblem
Пусть в~тетраэдре $SABC$ точка~$O$~--- центр описанной сферы.
Пусть $A'$, $B'$ и~$C'$~--- образы точек $A$, $B$ и~$C$ соответственно при
некоторой инверсии с~центром в~$S$.
Докажите, что $SO$~--- прямая, содержащая высоту в~тетраэдре $SA'B'C'$.

\end{problems}


\subsection*{Задачи}

\begin{problems}

\item
\subproblem
Прямые $a$ и~$b$ пересекаются в~точке~$L$.
Прямая~$a$ пересекает окружность~$\omega$ в~точках $A_1$ и~$A_2$,
а~$b$ пересекает $\omega$ в~точках $B_1$ и~$B_2$.
Пусть $R_1$~--- центр описанной окружности в~треугольнике $L A_1 B_1$.
Докажите, что $LR$~--- высота в~треугольнике $L A_2 B_2$.
\\
\subproblem
В~условиях предыдущего пункта пусть $O$~--- центр окружности~$\omega$.
Докажите, что $L R_1 = O R_2$, где $R_2$ определяется аналогично $R_1$.

\item
Дан остроугольный треугольник $ABC$, причем $AB > AC$
и~$\angle BAC = 60^{\circ}$.
Пусть $O$~--- центр описанной окружности треугольника, а~$H$~--- ортоцентр.
Прямая $OH$ пересекает стороны $AB$ и~$AC$ в~точках $P$ и~$Q$ соответственно.
Найдите отношение $PO : HQ$.
% Многоборье 2015

\item
Найдите углы остроугольного треугольника $ABC$, если известно, что его
биссектриса~$AD$ равна стороне~$AC$ и~перпендикулярна отрезку~$OH$, где $O$~---
центр описанной окружности, $H$~--- точка пересечения высот треугольника $ABC$.

\item
В~трапеции $ABCD$ с~основаниями $AB$ и~$CD$ выполнено $AB = 2 CD$.
Обозначим прямую, перепендикулярную $CD$ и~проходящую через точку~$C$, через
$l$.
Окружность с~центром $D$ и~радиусом $DA$ пересекает $l$ в~точках $P$ и~$Q$.
Докажите, что $AP$ перпендикулярна $BQ$.
% сентябрьская олимпиада СУНЦ 2015

\end{problems}

\endgroup % \def\vect

