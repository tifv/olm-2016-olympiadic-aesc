% $date: 2015-10-07 # среда

\section*{Индукция}

% $authors:
% - Виктор Трещёв

\begin{problems}

\item
Проведем в~выпуклом многоугольнике некоторые диагонали так, что никакие две
из~них не~пересекаются (из~одной вершины могут выходить несколько диагоналей).
Доказать, что найдутся по~крайней мере две вершины многоугольника, из~которых
не~проведено ни~одной диагонали.

\item
В~квадрате $1024 \times 1024$ вырезали одну клетку.
Докажите, что оставшееся можно разрезать на~уголки из~трех клеток.

\item
На~какую максимальную степень тройки делится число, десятичная запись которого
состоит из~$3^n$ единиц?

\item
Петя Торт умеет на~любом отрезке отмечать точки, которые делят этот отрезок
пополам или в~отношении $n : (n + 1)$, где $n$~--- любое натуральное число.
Петя утверждает, что этого достаточно, чтобы разделить отрезок на~любое
количество одинаковых частей.
Прав~ли он?

\item
В~компании из $k$~человек ($k > 3$) у~каждого появилась новость, известная ему
одному.
За~один телефонный разговор двое сообщают друг другу все известные им новости.
Докажите, что за~$(2 k - 4)$ разговора все они могут узнать все новости.

\item
Даны натуральные числа $x_1, \dots, x_n$.
Докажите, что число
\(
    ( 1 + x_1^2 ) \cdot
    ( 1 + x_2^2 ) \cdot
    \ldots \cdot
    ( 1 + x_n^2 )
\)
можно представить в~виде суммы квадратов двух целых чисел.

\item
Определим числа $K_n$:
$K_0 = 1$ и~$K_{n + 1} = 1 + \min (2 K_{[n / 2]}, 3K_{[n / 3]})$.
Докажите, что $K_n \geq n$.

\item
Из~чисел от~$1$ до~$2n$ выбрано $n + 1$ число.
Докажите, что среди выбранных чисел найдутся два, одно из~которых делится
на~другое.

\item
Доказать, что по~окончании волейбольного турнира с~участием $2^n$ команд
(в~один круг) можно выбрать команды $K_1, K_2, \dots, K_{n+1}$ так, что каждая
из~команд $K_j$, $j \leq n$, выиграла у~всех команд
$K_{j+1}, K_{j+2}, \ldots, K_n$.

\item
В~$n$~мензурок налиты $n$ разных жидкостей, кроме того, имеется одна пустая
мензурка.
Можно~ли за~конечное число операций составить равномерные смеси в~каждой
мензурке, то~есть сделать так, чтобы в~каждой мензурке было равно $1 / n$
от~начального количества каждой жидкости, и~при этом одна мензурка была~бы
пустой.
(Мензурки одинаковые, но~количества жидкостей в~них могут быть разными;
предполагается, что можно отмерять любой объем жидкости.)

\item
Есть четное число комнат, в~каждой по~три лампочки.
Лампочки разбиты на~пары (в~паре могут быть лампочки из~разных комнат).
На~каждую пару по~одному выключателю, он при нажатии меняет состояние обеих
лампочек в~паре на~противоположное.
Докажите, что вне зависимости от~того, какие лампочки горели в~начале, можно
сделать так, чтобы в~каждой комнате хотя~бы одна лампочка горела и~хотя~бы одна
не~горела.

\end{problems}

