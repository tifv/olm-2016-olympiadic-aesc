% $date: 2016-03-02 # среда

\section*{Комбинаторная геометрия. Добавка}

% $authors:
% - Виктор Трещёв

\begin{problems}

\item
Cуществует~ли шестиугольник, который можно разбить одной прямой на~четыре
равных треугольника?

\item
На~плоскости отмечены $100$ точек, никакие три из~которых не~лежат на~одной
прямой.
Саша разбивает точки на~пары, после чего соединяет точки в~каждой из~пар
отрезком.
Всегда~ли он может это сделать так, чтобы каждые два отрезка пересекались?

\item
Можно~ли так раскрасить все клетки бесконечной клетчатой плоскости в~черный
и~белый цвета, чтобы каждая вертикальная прямая и~каждая горизонтальная прямая
пересекали пересекали конечное число белых клеток, а~каждая наклонная
прямая~--- конечное число черных?

\item
Про бесконечный набор прямоугольников известно, что в~нем для любого числа~$S$
найдутся прямоугольники суммарной площади больше $S$.
\\
\subproblem
Обязательно~ли этим набором можно покрыть всю плоскость, если при этом
допускаются наложения?
\\
\subproblem
Тот~же вопрос, если дополнительно известно, что все прямоугольники в~наборе
являются квадратами.

\end{problems}

