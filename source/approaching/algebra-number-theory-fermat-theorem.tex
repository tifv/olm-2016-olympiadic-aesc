% $date: 2015-11-20 # среда

\section*{Малая теорема Ферма}

% $authors:
% - Виктор Трещёв

\begin{problems}

\item
Докажите, что для любого целого числа $a$ число $a^{561} - a$ делится на~$561$.

\item
Известно, что число $a^{12} + b^{12} + c^{12} + d^{12} + e^{12} + f^{12}$
делится на~$13$.
Докажите, что $a b c d e f$ делится на~$13^{6}$.

\item
Докажите, что если $x^2 + 1$ делится на~нечетное простое $p$, то~$p = 4k + 1$.
При помощи этого докажите, что существует бесконечно много простых чисел вида
$p = 4k + 1$.

\item
Дано простое $p$ и~целое $a$, не~делящееся на~$p$.
Пусть $k$~--- наименьшее натуральное число такое, что $a^k \equiv 1 \pmod{p}$.
Докажите, что $(p - 1)$ делится на~$k$.

\item
Докажите, что число $30^{239} + 239^{30}$~--- составное.

\item
Пусть для простого числа $p > 2$ и~целого $a$, не~делящегося на~$p$,
выполнено сравнение $x^2 \equiv a \pmod p$.
Докажите, что $a^{(p-1)/2} \equiv 1 \pmod p$.

\item
Докажите, что если $p$~--- простое число, $p \neq 2, 5$, то~длина периода
разложения $1 / p$ в~десятичную дробь делит $(p - 1)$.

\item
Даны натуральные $x, y \in [2; 100]$.
Докажите, что при некотором натуральном~$n$ число $x^{2^n} + y^{2^n}$ составное.

\end{problems}

