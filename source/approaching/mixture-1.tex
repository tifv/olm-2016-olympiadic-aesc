% $date: 2016-03-09 # среда

\section*{Разнобой}

% $authors:
% - Глеб Погудин

\begin{problems}

\item
Решите в~целых числах уравнение $3^x = 7 x^2 - 6 y^3$.

\item
Алёша написал на~доске пять целых чисел~--- коэффициенты и~корни квадратного
трёхчлена.
Боря стер одно из~них.
Остались числа $2$, $3$, $4$, $-5$.
Восстановите стертое число.

\item
Дан выпуклый шестиугольник $ABCDEF$.
Известно, что $\angle FAE = \angle BDC$, а~четырехугольники $ABDF$ и~$ACDE$
являются вписанными.
Докажите, что прямые $BF$ и~$CE$ параллельны.

\item
Какое наименьшее количество трехклеточных уголков можно разместить в~квадрате
$8 \times 8$ так, чтобы в~этот квадрат больше нельзя было поместить ни~одного
такого уголка?

\item
Для всякого положительного $x$ докажите неравенство
$2 x + 3 / 8 \geq \sqrt[4]{x}$.

\item
Гидры состоят из~голов и~шей (каждая шея соединяет ровно две головы).
Одним ударом меча можно снести все шеи, выходящие из~какой-то головы $A$ гидры.
Но~при этом из~головы A мгновенно вырастает по~одной шее во~все головы,
с~которыми A не~была соединена.
Геракл побеждает гидру, если ему удастся разрубить ее на~две несвязанные шеями
части.
Найдите наименьшее $N$, при котором Геракл сможет победить любую стошеюю гидру,
нанеся не~более, чем $N$~ударов.

\end{problems}

